\section{Annexe 3 - Bibliographie}

ADEME (2022). Déchets chiffres-clés – Édition 2022. Agence de la Transition Écologique.\\

Barraqué, B., & Larrue, C. (2021). Justice environnementale et politiques locales de l'environnement. L'Harmattan.\\

CITEPA (2023). Inventaire des émissions de gaz à effet de serre de la France. Centre interprofessionnel technique d'études de la pollution atmosphérique.\\

Haut Conseil pour le Climat (2022). Maîtriser l’empreinte carbone de la France : une ambition climatique pour tous. Rapport annuel.\\

IGEDD – Inspection Générale de l'Environnement et du Développement Durable (2022). Évaluation de la mise en œuvre des procédures d’autorisation environnementale. Ministère de la Transition écologique.\\

Ministère de la Transition Écologique (2023). Bilan des incendies sur les installations de déchets et ICPE en France. Direction Générale de la Prévention des Risques (DGPR).\\

Moulis, C. (2019). Les installations classées pour la protection de l’environnement : un droit à la croisée des chemins. Revue juridique de l’environnement, 44(2).\\

Code de l’environnement, Partie législative et réglementaire, Livre V : Prévention des pollutions, des risques et des nuisances.
Éditions Dalloz, mise à jour annuelle.
Référence pour les articles L511-1 à L517-2.\\

Ministère de la Transition écologique – Direction Générale de la Prévention des Risques (DGPR) (2023).
Chiffres clés des ICPE en France.
Paris, Ministère de la Transition écologique.\\

IGEDD (Inspection Générale de l’Environnement et du Développement Durable) (2022).
Évaluation du régime d’autorisation environnementale.
La Documentation Française. Rapport public.\\

INERIS (Institut National de l’Environnement Industriel et des Risques) (2021).
Cadre réglementaire des installations classées : approche par les risques industriels.
Rapport technique, Verneuil-en-Halatte.\\

ADEME (2021). Chiffres-clés des déchets – Édition 2021. Agence de la transition écologique.\\

France Stratégie (2021). Souveraineté industrielle et économie circulaire : défis et opportunités. Paris : La Documentation Française.\\

Coutard, O., Lorrain, D. (2011). Les réseaux territoriaux de l'économie circulaire. Revue d’Économie Régionale et Urbaine, n°5.\\

Le Guen, R., (2019). Inégalités socio-environnementales en France : état des lieux et perspectives. Commissariat Général au Développement Durable (CGDD).\\

Durand, A. (2020). Justice environnementale et gouvernance territoriale des déchets. Paris : L’Harmattan.\\

Laurent, É. (2011). Le nouvel âge de l’inégalité : économie et politique dans une société en transition. Éditions du Seuil.\\

Barles, S. (2007). Urban metabolism and recycling policies in Paris: History and current issues. Journal of Industrial Ecology, 11(2), 5–15.\\

CIREGE (2020). La territorialisation des enjeux de l’économie circulaire : une lecture critique. Université de Reims.\\

Vernier, J (1971). La bataille de l'environnement\\

Tzu, S (-500 env). L'art de la Guerre\\

Faure, E (1966). Prévoir le présent\\