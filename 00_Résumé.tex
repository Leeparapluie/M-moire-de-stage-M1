\section{Résumé}

Le système des Installations Classées pour la Protection de l’Environnement (\gls{ICPE}) réglemente en France les activités industrielles, agricoles ou artisanales qui peuvent causer de la pollution, nuire ou présenter un risque pour la santé humaine, la sécurité publique et l'environnement. Cet ensemble de règles aide à concevoir des mesures techniques et législatives appropriées pour atténuer les défis environnementaux et sociologiques posés par de telles installations.\\

Même si cette réglementation est destinée à gérer le risque industriel tout en protégeant les populations et les écosystèmes à long terme, elle présente des lacunes flagrantes dans son application réelle. Plus important encore, l'uniformité des textes réglementaires entre en conflit avec la diversité des contextes sociaux locaux dans lesquels les projets sont situés. Une telle hétérogénéité territoriale peut conduire à une forme de distribution inégale des avantages et des charges environnementales, par exemple en ce qui concerne les inégalités autour des installations de gestion des déchets.\\

Ces installations de collecte des déchets semblent offrir une opportunité intéressante pour réduire notre dépendance aux marchés étrangers et à la consommation de ressources naturelles tout en résolvant l'un des défis les plus anciens de l'humanité, mais il existe encore de nombreux obstacles à surmonter pour leur plein développement.\\

