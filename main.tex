\documentclass[12pt, a4paper]{article}

\usepackage{fancyhdr}
\usepackage{lastpage}
\usepackage{a4wide} 
\usepackage{amsmath}
\usepackage{amssymb} 
\usepackage{graphicx}
\usepackage{color}
\usepackage{fancybox}
\usepackage{moreverb}
\usepackage{hyperref}
\usepackage{array}
\usepackage{float}
\usepackage{glossaries}
\usepackage{subfig}
\usepackage{tabularx}
\usepackage{multirow}
\usepackage{wrapfig}
\usepackage[table]{xcolor}
\usepackage[toc,page]{appendix} 

\newcounter{tblerows}
\expandafter\let\csname c@tblerows\endcsname\rownum

\definecolor{table_1_color}{RGB}{255, 255, 255}
\definecolor{table_2_color}{RGB}{245, 245, 245}

\setlength{\tabcolsep}{8pt}
\renewcommand{\arraystretch}{1.5}

\renewcommand{\appendixpagename}{Annexes}
\renewcommand{\appendixtocname}{Annexes}

% \renewcommand{\thesection}{\Alph{section}} 
% \renewcommand{\thesubsection}{\thesection.\alph{subsection}}

\title{Mémoire de stage}
\author{CRETEUR Tom}
\date{Juin 2025}

\makeglossaries

\newglossaryentry{ADEME}
{
    name=ADEME,
    description={Agence de la transition écologique, anciennement Agence de l'environnement et de la maîtrise de l'énergie / établissement public à caractère industriel et commercial (EPIC) français dont la vocation est de susciter, animer, coordonner, faciliter ou réaliser des opérations de protection de l'environnement}
}
\newglossaryentry{AEE}
{
    name=AEE,
    description={Agence européenne pour l'environnement (AEE) / agence de l'Union européenne qui fournit des informations et des données à l'appui des objectifs européens en matière d'environnement et de climat}
}
\newglossaryentry{AGEC}
{
    name=AGEC,
    description={(loi) Anti-Gaspillage pour une Economie Circulaire}
}
\newglossaryentry{AIDA}
{
    name=AIDA,
    description={Site français d’information relatif au droit de l’environnement}
}
\newglossaryentry{BRGM}
{
    name=BRGM,
    description={Bureau de Recherches Géologiques et Minières}
}
\newglossaryentry{CEREMA}
{
    name=CEREMA,
    description={Centre d’Etudes et d’Expertise sur les Risques, l’Environnement, la Mobilité et l’Aménagement}
}
\newglossaryentry{Cerfa}
{
    name=Cerfa,
    description={Il s'agit de tous les documents, quels qu'en soient la présentation et le support, y compris électronique, permettant à un usager d'accomplir une démarche administrative}
}
\newglossaryentry{DGPR}
{
    name=DGPR,
    description={Direction générale de la prévention des risques}
}
\newglossaryentry{DREAL}
{
    name=DREAL,
    description={Direction Régionale de l’Environnement, de l’Aménagement et du Logement}
}
\newglossaryentry{ERC}
{
    name=ERC,
    description={Eviter, Réduire, Compenser (séquence)}
}
\newglossaryentry{ERP}
{
    name=ERP,
    description={Etablissement Recevant du Public}
}
\newglossaryentry{ICPE}
{
    name=ICPE,
    description={Installation Classée pour la Protection de l'Environnement}
}
\newglossaryentry{IGN}
{
    name=IGN,
    description={Institut National de l’Information Géographique et Forestière}
}
\newglossaryentry{INERIS}
{
    name=INERIS,
    description={Institut National de l’Environnement Industriel et des Risques}
}
\newglossaryentry{INRS}
{
    name=INRS,
    description={Institut National de Recherche et de Sécurité}
}
\newglossaryentry{INSEE}
{
    name=INSEE,
    description={Institut National de la Statistique et des Etudes Economiques}
}
\newglossaryentry{IOTA}
{
    name=IOTA,
    description={Installations Ouvrages Travaux et Activités}
}
\newglossaryentry{MTD}
{
    name=MTD,
    description={Meilleures Techniques Disponibles (principe)}
}
\newglossaryentry{OFB}
{
    name=OFB,
    description={Office Français de la Biodiversité}
}
\newglossaryentry{PAC}
{
    name=PAC,
    description={Porter à connaissance}
}
\newglossaryentry{PLU}
{
    name=PLU,
    description={Plan local d’Urbanisme}
}
\newglossaryentry{PNM}
{
    name=PNM,
    description={Parc Naturel Marin}
}
\newglossaryentry{PNR}
{
    name=PNR,
    description={Parc Naturel Régional}
}
\newglossaryentry{PPRCS}
{
    name=PPRCS,
    description={Plan de Prévention des Risques Cavités Souterraines}
}
\newglossaryentry{PPRi}
{
    name=PPRi,
    description={Plan de Prévention des Risques Inondations}
}
\newglossaryentry{PPRMT}
{
    name=PPRMT,
    description={Plan de Prévention des Risques Mouvements de Terrain}
}
\newglossaryentry{PPRS}
{
    name=PPRS,
    description={Plan de Prévention des Risques Sismiques}
}
\newglossaryentry{PPRT}
{
    name=PPRT,
    description={Plan de Prévention des Risques Technologiques}
}
\newglossaryentry{SCoT}
{
    name=SCoT,
    description={Schéma de Cohérence Territoriale}
}
\newglossaryentry{SAGE}
{
    name=SAGE,
    description={Schéma d’Aménagement et de Gestion des Eaux}
}
\newglossaryentry{SDAGE}
{
    name=SDAGE,
    description={Schéma Directeur d’Aménagement et de Gestion des Eaux}
}
\newglossaryentry{SDIS}
{
    name=SDIS,
    description={Service Départemental d’Incendie et de Secours}
}
\newglossaryentry{SUP}
{
    name=SUP,
    description={Servitudes d’Utilité Publique}
}
\newglossaryentry{ZAN}
{
    name=ZAN,
    description={Zéro Artificialisation Nette}
}
\newglossaryentry{ZNIEFF}
{
    name=ZNIEFF,
    description={Zones Naturelles d’Intérêt Ecologique, Faunistique et Floristique}
}


\begin{document}

    \thispagestyle{empty}
    
    \begin{center}
        \makebox[\textwidth][l]{
            \raisebox{0pt}[0pt][0pt]{
                \includegraphics[scale=1]{Images/Evolutys.jpg}
            }
        }
        \makebox[\textwidth][r]{
            \raisebox{0pt}[0pt][0pt]{
                \includegraphics[scale=1]{Images/Université.png}
            }
        }
    
        Université Jean Monnet Saint-Etienne\\
        Master Géographie, Aménagement, Environnement et Développement\\
        \vspace{3cm}
        {\LARGE Mémoire de stage}\\
        \vspace{1cm}
        Effectué chez Evolutys \\
        \vspace{1cm}
        \shadowbox{
            \begin{minipage}{1\textwidth}
                \begin{center}
                    {\Huge Réglementation des Installations Classées pour la Protection de l’Environnement (ICPE) et enjeux socio-environnementaux. L’exemple de la gestion des déchets et du développement des infrastructures dédiées.}\\
                \end{center}
            \end{minipage}
        }\\
        \vspace{2cm}
        Créteur Tom\\
        1e année -- Parcours SENTINELLES (Santé ENvironnemenT INformations spatio-temporELLES) \\
        \vspace{3mm}
        10 mars 2025 -- 30 juillet 2025 \\
        \vspace{2cm}
        {\renewcommand{\arraystretch}{1}
            \begin{tabular}{p{10cm}p{10cm}}
                {\bf Evolutys}                                            &{\bf Responsable de stage}\\
                {\footnotesize 4 bis rue du Garat }       & ~~~Magrin Franck\\
                {\footnotesize 42152 L’Horme}                                        & {\bf Tutrice pédagogique}\\
                {\footnotesize }                          & ~~~Chasles Virginie\\
            \end{tabular}
        }
    \end{center}
    \newpage
        
    \tableofcontents
    \listoffigures

    \newpage
    \printglossaries        

    \section{Remerciements}


Je souhaite tout d’abord exprimer ma sincère gratitude à Monsieur Philippe Gasquet, pour m’avoir offert l’opportunité d’intégrer l’entreprise EVOLUTYS dans le cadre de ce stage.\\

Mes remerciements vont également à Monsieur Franck Magrin, mon tuteur de stage, pour son accompagnement précieux tout au long de ce stage, malgré les aléas rencontrés. J’adresse également mes remerciements à Madame Virginie Chasles, ma responsable de stage.\\

Je tiens à remercier chaleureusement Madame Marie Granger, Madame Océane Jacques et Madame Estelle Cambon, pour leur accueil, leur bienveillance et leur exigence professionnelle. Leur disponibilité m’a permis de comprendre et d’appliquer avec rigueur les exigences relatives à la réalisation de dossiers réglementaires et d’audits ICPE. Aux côtés de Théo, mon camarade d'audit, j’ai pu apprendre bien au-delà de mes attentes.\\

Je souhaite à l’ensemble de l’équipe d’EVOLUTYS une excellente continuation.\\

Je remercie également ma famille pour son soutien constant et indéfectible tout au long de mon parcours universitaire.\\

Enfin, je tiens à adresser une pensée particulière à mes amis et camarades qui, dans des circonstances parfois complexes, m’ont apporté un soutien moral précieux : Vivier, "le petit Berger", Tao l’ingénieur informaticien, les Paul, Clarisse, Louise, Andréa, Laurine, Anna, Maya, Marie, Lola, Quentin, sans oublier la fine équipe des Crous de la Tronche et du Rabot, Adrien, ainsi que l’ensemble de mes camarades de promotion. À tous, merci pour votre présence et votre amitié.\\
 
    \section{Résumé}

Le système des Installations Classées pour la Protection de l’Environnement (\gls{ICPE}) réglemente en France les activités industrielles, agricoles ou artisanales qui peuvent causer de la pollution, nuire ou présenter un risque pour la santé humaine, la sécurité publique et l'environnement. Cet ensemble de règles aide à concevoir des mesures techniques et législatives appropriées pour atténuer les défis environnementaux et sociologiques posés par de telles installations.\\

Même si cette réglementation est destinée à gérer le risque industriel tout en protégeant les populations et les écosystèmes à long terme, elle présente des lacunes flagrantes dans son application réelle. Plus important encore, l'uniformité des textes réglementaires entre en conflit avec la diversité des contextes sociaux locaux dans lesquels les projets sont situés. Une telle hétérogénéité territoriale peut conduire à une forme de distribution inégale des avantages et des charges environnementales, par exemple en ce qui concerne les inégalités autour des installations de gestion des déchets.\\

Ces installations de collecte des déchets semblent offrir une opportunité intéressante pour réduire notre dépendance aux marchés étrangers et à la consommation de ressources naturelles tout en résolvant l'un des défis les plus anciens de l'humanité, mais il existe encore de nombreux obstacles à surmonter pour leur plein développement.\\


    \newpage
\sloppy
\section{Introduction}

Qu'est-ce qu'un déchet ?\\

Jacques Vernier, dans son livre La bataille de l’environnement (1971), le définit comme :\\
\begin{center}
"Un produit dont personne ne veut à l’endroit où il se trouve."
\end{center}
C’est souvent un objet nuisible pour la vue, l’odorat ou la santé, et que la nature refuse de réintroduire dans son cycle.\\
Pour le préhistorien ou l’archéologue, le déchet est un trésor :
Un vase ébréché, le squelette d’un animal vieux de milliers d’années, un outil rouillé de l’époque sumérienne sont une mine d’informations pour la science.\\

Les chasseurs de manuscrits trouvent la majorité de leurs documents originaux en fouillant dans les déchets produits par les auteurs célèbres. C’est en apprenant que sa propre poubelle faisait régulièrement l’objet de telles investigations que Jean-Paul Sartre prit la décision radicale de détruire systématiquement tous les brouillons de ses écrits en les brûlant.\\

Le déchet, en tant que trace matérielle de l’activité humaine, est aussi ancien que les sociétés elles-mêmes. Dans le contexte contemporain, marqué par la surconsommation et la production massive de biens, il acquiert une signification nouvelle. Il peut, si l’on parvient à en tirer pleinement parti, devenir une ressource stratégique. Ainsi, les rebuts de la société de consommation pourraient-ils se transformer en véritables "cornes d’abondance ", pour peu que les conditions techniques, économiques et sociales soient réunies. Néanmoins, pour que ce potentiel se réalise, les installations dédiées à la gestion des déchets devront surmonter un ensemble de défis majeurs. Parmi ceux-ci figurent notamment l’innovation technologique, la mobilisation d’investissements conséquents, ainsi que la résilience face aux risques, y compris ceux liés à des catastrophes industrielles ou environnementales.\\

Le lundi 7 avril, à environ 19h45, un incendie a vite pris dans le centre de tri des collectes sélectives du Syctom, sis dans le 17e arrondissement de Paris. Malgré l’intervention rapide des agents d’exploitation et l’activation des dispositifs de sécurité opérationnels, les premières mesures n’ont pas permis de maîtriser le sinistre, occasionnant des dommages matériels conséquents. 

     \begin{figure}[H]
            \includegraphics[width=\textwidth]{Images/Incendie.jpg}
            \centering
            \caption{Incendie du centre de tri des collectes sélectives du Syctom : Face au Risque 7 avril 2025}
        \end{figure}


Ce type d’événement illustre les aléas inhérents à l’exploitation de telles installations vouées à une gestion des déchets. Ce n’est rien d’autre qu’une illustration de questions plus fondamentales qui relèvent de la sécurité industrielle, de la prévention des pollutions mais aussi de la maîtrise des nuisances. Ces installations, déclarées au titre des Installations Classées pour la Protection de l’Environnement (\gls{ICPE}) manipulent au quotidien des matières et substances dont le caractère pourrait être inflammable, toxique ou polluant.\\

En France, ce sont près de 500 centres de tri et plateformes de recyclage qui relèvent du régime \gls{ICPE}, compte tenu de la nature des matériaux traités et des impacts environnementaux potentiellement associés à leurs activités (\gls{ADEME}). La fréquence et la gravité des incidents rapportés soulignent l’importance d’une régulation rigoureuse, fondée sur la prévention des risques, le contrôle des installations et l’amélioration continue des pratiques dans le secteur du traitement et de la valorisation des déchets.\\

La réglementation relative aux \gls{ICPE}, instaurée par la loi du 19 juillet 1976, a pour objectif principal la prévention des nuisances et des risques susceptibles d’être générés par certaines activités industrielles et agricoles. Elle repose sur un encadrement administratif rigoureux, modulé en fonction du niveau de danger que présentent les installations concernées. Ce dispositif réglementaire s’inscrit aujourd’hui dans un contexte de transition vers une économie circulaire, qui constitue à la fois une priorité nationale et un engagement fort au niveau européen. Dans cette perspective, les infrastructures de recyclage occupent une position stratégique : elles contribuent à la réduction de l’enfouissement des déchets, à la limitation de la consommation de ressources primaires, ainsi qu’à la diminution des émissions de gaz à effet de serre.\\

Néanmoins, le développement de ces infrastructures soulève des enjeux sociaux et territoriaux importants. Parmi ceux-ci figurent l’opposition des riverains, la tendance à concentrer les installations dans des zones déjà marquées par des vulnérabilités socio-économiques, le déficit d’acceptabilité locale et l’inégalité d’accès aux bénéfices environnementaux générés par ces dispositifs. Ces problématiques témoignent de la complexité à concilier les impératifs environnementaux avec les dynamiques sociales et spatiales dans les territoires.\\

Dans ce contexte, alors que la France s’est engagée, par le biais de la loi relative à la lutte contre le gaspillage et à l’économie circulaire (loi \gls{AGEC}, 2020), à accroître significativement le nombre d’installations de valorisation matière pour atteindre un objectif de 65 \% de recyclage des déchets municipaux d’ici à 2035, le cadre réglementaire ICPE apparaît comme un outil ambivalent. D’une part, il constitue un levier essentiel pour sécuriser le développement de ces activités du point de vue environnemental. D’autre part, il met en lumière les tensions persistantes qui entourent l’implantation territoriale de ces infrastructures. Comme le souligne l’Agence Européenne pour l’Environnement (\gls{AEE}, 2022), les activités industrielles, y compris celles considérées comme « vertes », continuent de générer des formes de pollution localisée et des nuisances, touchant de manière disproportionnée les populations les plus vulnérables.\\

Ce rapport a pour but d’analyser de façon approfondie comment le cadre réglementaire des \gls{ICPE} encadre le développement des installations de gestion des déchets et, donc, dans quelle mesure ce cadre permet de prévenir et de circonscrire les impacts environnementaux et sociaux potentiellement engendrés par ces activités. La réglementation ICPE comme outil de prévention des risques industriel et environnemental est un dispositif clé au pilotage du développement de ces installations potentiellement impactantes.\\

La problématique est donc la suivante :\\

\textit{Comment la réglementation des ICPE permet-elle d’encadrer efficacement le développement des infrastructures de gestion des déchets, tout en limitant leurs impacts environnementaux et sociaux ?}\\

Dans un premier temps, nous reviendrons sur le rôle structurant de la réglementation \gls{ICPE} dans la gestion des risques inhérents aux activités de traitement et de valorisation des déchets, et nous verrons de quelle façon ce cadre réglementaire basé sur une approche fondée sur des principes de prévention, de précaution et de proportionnalité va encadrer le recyclage pour en limiter les effets nocifs pour la santé humaine et pour l’environnement. Nous prendrons également en compte les exigences techniques et organisationnelles afférentes aux exploitants au sein de ce cadre réglementaire, ainsi que les outils de contrôle et de rédaction mis en oeuvre par les administrations chargées de l’autorisation et du contrôle.\\

Dans une seconde analyse, nous nous interrogerons sur l’instruction et la constitution dans son ensemble d’un dossier \gls{ICPE} pour un projet de création ou de modification d’installation de recyclage. Nous mettrons ici en exergue les principales phases administratives et techniques, c’est-à-dire de la recherche du site jusqu’à l’obtention de l’autorisation ou à l’enregistrement, en passant par l’élaboration de l’étude d’impact, la concertation du public et l’évaluation des risques. L’objectif est ici de mettre en évidence une certaine complexité dans le processus ainsi qu’un certain nombre d’exigences spécifiques pour les porteurs de projet dans ce secteur.\\

Enfin, nous nous pencherons sur les enjeux socio-environnementaux liés au développement de ces installations sur le territoire. Nous analyserons en particulier les disparités d’exposition des populations aux nuisances générées par ces activités (bruit, odeurs, trafic, pollution, etc.) et les facteurs susceptibles de renforcer ou de limiter leur acceptabilité sociale. Cette dernière partie permettra d’éclairer les tensions potentielles entre impératifs écologiques, équité territoriale et dynamiques locales d’aménagement.\\

    \newpage
\sloppy
\section{ La réglementation \gls{ICPE} face aux enjeux environnementaux des activités de gestion des déchets }
    
    \subsection{Les défis environnementaux posés par la gestion et le traitement des déchets }
    
    La préservation de l’environnement contribue en grande partie au processus de la transition écologique. De ce fait, la gestion des déchets s’avère être une question cruciale. En France, l’\gls{ADEME} souligne qu’au niveau national, la production annuelle excède 310 millions de tonnes de déchets par an, tous secteurs confondus. D’entre les divers secteurs d’activité, les ménages s’avèrent en outre en tête avec près de 40 millions par an. Le traitement de cette masse considérable engendre énormément de défis écologiques sur le plan environnemental et doit relever plusieurs défis à chaque étape de la gestion des déchets. Au regard des chiffres fournis par la Banque mondiale concernant d’autres pays, on remarque que d’ici 2100 la production mondiale de déchets pourrait tripler par rapport à celle de 2010.\\

     \begin{figure}[H]
            \includegraphics[width=\textwidth]{Images/Banque mondiale.png}
            \centering
            \caption{Graphique de la production de déchets projetée 2010-2100 par régions de la planète : Marcelo Pires Negrao 2017}
    \end{figure}
        
     Les installations de gestion de déchets sont une source considérable de pollution. Les ISDND (installations de stockage de déchets non dangereux) par exemple, à l'enfouissement des déchets, produisent des lixiviats qui contiennent des liquides toxiques. Leur toxicité dépend du traitement qu’ils reçoivent ainsi que leur collecte. En outre, les centres qui traitent ou valorisent émettent aussi des composés organiques volatils, de la poussière ou autres particules nocives pour la santé. Par ailleurs, il existe un risque bien plus dangereux que ceux provoqués par ces rejets. Avec le temps, ils pourraient causer un grand nombre d'incendies.\\
     
     À l’échelle climatique, le secteur des déchets représente environ 3 pourcents des émissions de gaz à effet de serre en France, selon le Citepa. Cette contribution est notamment liée à la production de méthane dans les décharges (lors de la décomposition des déchets organiques), un gaz au pouvoir de réchauffement global 25 fois supérieur à celui du }CO_2$.
     Les incinérateurs, eux, par exemple émettent du dioxyde de carbone, notamment lors de la combustion de plastiques issus de produits non recyclés.  
     

      \begin{figure}[H]
            \includegraphics[width=\textwidth]{Images/Gas effet de serre.jpg}
            \centering
            \caption{Emissions de gaz à effet de serre par secteur en France sur la période 1990-2013 : Ministère de l'amènagement du territoire et de la transition écologique 2015}
        \end{figure}
     
     En France, les moyens déployés pour traiter nos ordures sont parfois à bout de souffle. Dans certains secteurs comme l’Île-de-France, les quantités de déchets à traiter arrivent à la capacité des usines, tandis que la rareté du foncier nuit à la construction de nouveaux équipements. Cette dureté s’amplifie avec la directive cadre européenne sur les déchets qui impose de diviser par 4 ou 5 le volume de déchets mis en décharge : il devra en représenter moins de 10 pourcents des déchets des ménages d’ici 2035. Il faut revoir toute la chaîne des déchets en faveur du recyclage et du réemploi.\\
     
     Le recyclage a grandement progressé, mais le rendement demeure faible dans certains filières. En effet, les matériaux recyclables sont trop souvent mal triés, contaminés ou complexes à traiter ; cela diminue le rendement comme la rentabilité des usines. L’économie circulaire ne saurait finalement fonctionner qu’avec une machine performante, à condition aussi de faire progresser la conception des produits pour optimiser leur réutilisation et leur recyclage. L’État tente de pousser dans ce sens avec la loi \gls{AGEC} (2020) qui impose de réduire les déchets plastiques à usage unique et durcit les consignes de tri avec des responsabilités élargies des producteurs mais les résultats sont à ce stade encore mitigés.\\
     
     Pour résumer, les défis environnementaux liés à la gestion des déchets sont : 
     \begin{itemize}
         \item contrôler la pollution,
         \item améliorer l'efficacité du recyclage,
         \item augmenter la capacité des installations
         \item sécuriser les installations,
         \item diminuer la quantité de déchets non valorisés,
     \end{itemize}    
     
     Avant d'explorer comment la réglementation \gls{ICPE} doit s'adapter à un cadre de gouvernance plus approprié aligné sur les objectifs et les exigences de la transition écologique ou quelles sont les fonctions d'un bureau d'études spécialisé dans les évaluations des \gls{ICPE}, examinons d'abord ce qu'est une \gls{ICPE}.
    
    \subsection{Nomenclature, rubriques et arrêtés ministériels : supports de base des \gls{ICPE}}

    Les Installations Classées pour la Protection de l'Environnement (\gls{ICPE}) ont été créées en 1976, cinq ans après l'accident industriel de Seveso, et se trouvent au livre V, titre I du Code de l’Environnement (articles L511-1 à L517-2). Ce règlement s'applique à toute usine, atelier, entrepôt, chantier, carrière et, plus largement, à toute installation, qu'elle soit exploitée ou détenue par une personne physique ou morale de droit privé ou public, qui pose des dangers ou des nuisances potentielles en termes de santé de la population riveraine, d'environnement sans violence, d'hygiène publique, d'agriculture, de conservation de la nature ainsi que de prévention des dommages à l'environnement et de durabilité.\\

        \subsubsection{La nomenclature \gls{ICPE} : une classification par types d’activités et dangers}
        
        La nomenclature \gls{ICPE}, régulièrement mise à jour par décret, dresse la liste des activités soumises à autorisation, enregistrement ou déclaration en fonction de la nature des substances manipulées ou des procédés industriels mis en œuvre. Le principe fondamental du régime \gls{ICPE} repose sur l’adaptation du niveau d’encadrement réglementaire en fonction du degré de dangerosité ou de nuisance potentielle de l’activité. Pour cela, une nomenclature \gls{ICPE}, composée d’environ 200 rubriques, classe les installations selon leur nature et leur niveau de risque. Cette nomenclature est organisée en quatre grandes catégories :

      \begin{figure}[H]
            \includegraphics[width=270]{Images/Structure.png}
            \centering
            \caption{Nomenclature \gls{ICPE} : \gls{AIDA} \gls{INERIS} 2020}
        \end{figure}

        Les rubriques concernant les déchets sont les rubriques 27xx mais lors de la mise en place d'une \gls{ICPE}, il est possible que certaines circonstances telles que l'utilisation de substances ou mélanges visés par la directive SEVESO ou encore l'installation d'un local de charge peuvent entraîner le classement d'un site dans plusieurs rubriques.
        Cette nomenclature est consultable sur Légifrance et constitue la pierre angulaire de toute procédure \gls{ICPE}. Elle permet à l’exploitant d’identifier précisément le régime applicable à son installation:

        \begin{itemize}
            \item \textbf{Déclaration} :  pour les activités à risques faibles
            \item \textbf{Enregistrement} :   pour les activités à risques modérés
            \item \textbf{Autorisation} :   pour les activités à risques forts
        \end{itemize}
        
        Selon les données de la \gls{DGPR}, la France comptait environ 500000 \gls{ICPE} dont plus de 20000 soumises à autorisation, illustrant la diversité et l’ampleur de ces installations sur le territoire.\\

        Le régime des (\gls{ICPE}) repose sur un principe fondamental : l’adaptation du niveau d’encadrement réglementaire au degré de danger ou de nuisance potentielle de l’activité. C’est pourquoi la nomenclature \gls{ICPE} distingue trois grands régimes : la déclaration, l’enregistrement et l’autorisation, chacun correspondant à un niveau d’exigence et de dangerosité croissant.
    
        \subsubsection{Le régime de déclaration : un encadrement allégé pour les activités à faibles risques}
        
        Le régime de déclaration, tel que défini par l’article L.512-8 du Code de l’environnement, s’applique aux installations dont les dangers ou les inconvénients pour l’environnement sont considérés comme limités. Ce régime constitue une procédure administrative simplifiée, visant à alléger les démarches imposées aux exploitants tout en assurant un niveau minimal de contrôle par l’administration.\\

        Dans ce cadre, l’exploitant n’est pas tenu d’obtenir une autorisation préalable avant de mettre en service son installation. Il doit toutefois accomplir une formalité déclarative préalable. Cette déclaration s’effectue soit par l’intermédiaire d’un téléservice dédié, dénommé TéléICPE, soit en remplissant le formulaire administratif spécifique intitulé Déclaration initiale d’une installation classée \gls{ICPE} (formulaire \gls{Cerfa} n°15271, version 15271*03).\\

        Une fois la déclaration transmise, l’exploitant peut débuter l’exploitation de son installation, à la condition de respecter strictement les prescriptions générales applicables à son activité. Ces prescriptions sont établies par arrêtés ministériels dits « arrêtés ministériels types », qui fixent les exigences techniques et environnementales de manière standardisée, en fonction de la nature de l’installation concernée.\\

        \begin{figure}[H]
            \includegraphics[width=350]{Images/Cerfa.png}
            \centering
            \caption{\gls{Cerfa} n°15271 (Déclaration initiale d'une installation classée) : Entreprendre Service Public 2025}
        \end{figure}
        
        Ce régime concerne par exemple certaines activités artisanales ou industrielles comme le stockage de produits non dangereux sous des seuils réglementaires (par exemple, des huiles ou solvants en petite quantité). En 2021, environ 400 000 installations relevaient de ce régime, selon les données de la Direction Générale de la Prévention des Risques (\gls{DGPR}) du Ministère de la Transition écologique.\\

        \subsubsection{Le régime d’enregistrement : un équilibre entre standardisation et contrôle}
        
        Institué par l'ordonnance du 11 juin 2009 et codifié à l'article L.512-7 du Code de l'environnement, le régime de l'enregistrement s'applique aux installations présentant des risques importants pour l'environnement, mais dont les derniers sont, en principe, maîtrisables, à condition que des prescriptions techniques normalisées soient strictement respectées. Le régime se place ainsi comme un mécanisme intermédiaire entre le régime de déclaration et celui de l'autorisation, reliant simplification administrative et exigence de protection environnementale forte.\\

        Le régime de l'enregistrement repose sur une démarche structurée, qui oblige l'exploitant à dresser et à transmettre un dossier technique complet. Le dossier devra porter la totalité des pièces dossier obligatoires en vue de la caractérisation exacte de l'installation prévue. Il couvre notamment la description de l'installation prévue et des futurs exploitants, description des nuisances potentielles pour le voisinage et pour l'environnement et les mesures techniques et les prescriptions spécifiques que l'exploitant s'engage à mettre en œuvre pour en limiter les effets.\\

        En parallèle, le fichier doit décrire les traits techniques des installations et équipements touchés, en indiquant leur compatibilité avec les documents d'urbanisme en place, à savoir les plans locaux d'urbanisme (\gls{PLU}) ou les schémas directeurs. Il doit aussi examiner les inconvénients et risques de l'exploitation de l'installation, en mesurant son impact potentiel à la fois sur l'environnement naturel et les populations riveraines. Enfin, l’exploitant est tenu de décrire les modalités de gestion, d’entretien et de suivi de l’installation, de manière à garantir la conformité permanente aux exigences réglementaires fixées par les arrêtés ministériels applicables au régime de l’enregistrement.

        
        \begin{itemize}
            \item \textbf{Pièce 1} : Type de demande - La première étape est une page de garde pour l’étude et permet d’annoncer le régime auquel est soumis le site.
            
            \item \textbf{Pièce 2} : Identification du demandeur -
            Cette étape identifie le mandataire du dossier et le mandant : Evolutys. Elle présente les informations de l’entreprise et contient une page de signature donnant au mandant l’autorisation de déposer le dossier en ligne au nom de l’entreprise mandataire.
            
            \item \textbf{Pièce 3} : Description du projet
            \begin{itemize}
            \item Description du projet - Le document contient également le détail des rubriques ICPE et IOTA concernées et la détermination du statut SEVESO ou non du site.
            \item Conformité arrêté ministériel - Ce document permet de justifier la conformité du projet à l’arrêté ministériel le concernant.
            \item Demande d’aménagement - En cas de non-respect d’un article de l’arrêté ministériel, une demande d’aménagement doit être faite pour justifier le fait de pouvoir y déroger. 
            \item Compatibilité PLU - Ce document permet de justifier la compatibilité du projet avec les différents documents d’urbanisme.
            \end{itemize}
            
            \item \textbf{Pièce 4} : Localisation - Cette étape donne la localisation géographique du site en coordonnées Lambert 93. Ces coordonnées sont déterminées par mesure sur l’outil Géoportail.
            
            \item \textbf{Pièce 5} : Activités - recense dans un tableau les rubriques ICPE par lesquelles le projet est concerné.
            
            \item \textbf{Pièce 6} : Impacts - Evaluation des incidences Natura 2000 / Incidences environnementales
            
            \item \textbf{Pièce 7} : Autres pièces 
            \begin{itemize}
            \item Capacités techniques et financières - Ce document a pour visée de prouver la capacité financière de l’entreprise à remettre son terrain en état en cas de cessation d’activité.
            \item Usage futur - Cette partie contient les documents du Maire et du propriétaire du terrain justifiant la remise en état du site en cas de cessation d’activité.
            \item Justificatif du dépôt de permis de construire
            \item Compatibilité avec les plans-schémas-programmes - Ce document contient l’ensemble des plans et schémas que doit respecter le site et la justification de compatibilité pour chaque point.
            \end{itemize}
            
            \item \textbf{Pièce 8} : Plans - Les plans sont en majorité réalisés par l’architecte engagé sur le projet. Les plans réalisés par le bureau d’études sont les plans de cadastre et IGN.
            
        \end{itemize}
        
        Ce régime est notamment applicable à de nombreuses installations de gestion de déchets, comme certaines installation de la rubrique 2710 (tri de déchets non dangereux).
        Les prescriptions sont fixées par arrêté ministériel (ex: arrêté du 26 mars 2012 modifié pour les rubriques 2710 à 2791) et doivent être appliquées strictement. Selon l’\gls{INERIS}, environ 20 000 installations sont aujourd’hui soumises au régime d'enregistrement en France.

        \subsubsection{Le régime d’autorisation : un contrôle renforcé pour les installations à hauts risques}
        
        La réglementation d'autorisation, telle que définie dans les articles L.512-1 et L.512-2 du Code de l'environnement, concerne les installations classées pour la protection de l'environnement (\gls{ICPE}) dont la dangerosité est considérée comme majeure. Il vise les projets susceptibles d'entraîner une atteinte à leur sens notable à la santé humaine, à la sécurité publique, à la qualité des eaux, de l'air ou des milieux naturels. Ce règlement est la méthode la plus requise en ce qui concerne le contrôle de l'environnement, du fait des dangers élevés que font courir les installations visées.\\

        La méthode d'autorisation impose une analyse exhaustive et approfondie du projet d'installation. Elle inclut, notamment, la production d'une étude d'impact sur l'environnement, à fin d'identifier, d'estimer et d'anticiper les effets potentiels de l'activité sur les milieux naturels et le cadre de vie. Pour les sites à haut risque, en particulier pour les sites classés Seveso, il est également obligatoire de faire une étude de danger pour apprécier les scenarios d'accident majeur et les voies de prévention et de maîtrise y afférents. Cette étape est suivie d'une enquête publique, par laquelle on associe la population et les parties prenantes locales à la formation du dossier. Le projet est soumis également à l'avis de l'Autorité environnementale et à l'avis technique des services de l'État concernés.\\

        La durée de la procédure peut être particulièrement significative, en se déployant généralement sur un temps de neuf à douze mois, voire plus lorsque survient une demande de pièces complémentaires ou de complexité exceptionnelle du dossier. À l'issue de cette instruction, le préfet a le pouvoir de délivrer un arrêté préfectoral d'autorisation. Ce décret fixe les règles de prescription définitive applicables à l'installation, qui peuvent excéder les règles générales, en fonction des caractéristiques intrinsèques du site, de son environnement et des risques identifiables. Un exemple typique d'installation soumise à ce régime est celui des incinérateurs de déchets, dont l'utilisation implique des risques élevés justifiant un encadrement règlementaire renforcé.\\

        Au sein même du régime d’autorisation, il convient d’opérer une distinction entre les installations classiques soumises à autorisation et celles qui relèvent de la directive Seveso, cette dernière étant transposée en droit français. Les sites dits « Seveso » constituent une catégorie à part au sein du régime d’autorisation, du fait de la présence ou de la manipulation de substances dangereuses dans des quantités qui dépassent certains seuils fixés par la norme européenne.\\

        En effet, ces installations Seveso ne constituent qu’un sous-ensemble du régime d’autorisation, c’est-à-dire que toutes les installations Seveso sont soumises au régime d’autorisation mais pas toutes les installations soumises à autorisation ne sont pas des sites Seveso, cette classification Seveso étant ensuite subdivisée par elle-même en deux catégories : les sites Seveso « seuil haut » qui présentent les risques les plus importants et qui sont soumis à des régimes de prévention des accidents majeurs, de sécurité industrielle et d’information du public plus rigoureux ; et ceux qui sont dits Seveso « seuil bas » qui disposent de moindres exigences mais plus rigoureuses que les installations sous le seul régime d’autorisation classique.\\

        La distinction découle ainsi essentiellement des catégories et de la quantité de substances dangereuses concernées, ainsi que des risques que celles-ci font peser sur les personnes, les biens et l’environnement. Si une installation sous le régime de l’autorisation ordinaire est déjà soumise à une expertise relative à ses impacts environnementaux et aux mesures de prévention des risques, une installation classée Séveso requiert, par ailleurs, de s’inscrire dans une politique de Prévention des Accidents Majeurs (PPAM), de disposer d’un Plan d’Opération Interne (POI) et, pour les installations dites à « seuil haut », d’un Plan Particulier d’Intervention (PPI) sous la conduite de l’autorité publique.\\

        À ces exigences s’ajoute un contrôle administratif renforcé, au niveau de la Direction régionale de l’environnement, de l’aménagement et du logement (DREAL). Les sites Seveso, notamment ceux classés « seuil haut », font en effet l’objet de visites régulières, en moyenne tous les un à trois ans, en fonction du risque engagé et de l’historique de l’établissement. Ces inspections visent à vérifier la conformité réglementaire, l’effectivité des mesures de prévention mises en œuvre ainsi que l’exercice des plans d’urgence. Un tel suivi avait pour but de garantir aux autorités un niveau maximal de sécurité industrielle puis de minimiser la probabilité ainsi que la gravité d’un accident majeur.\\


      \begin{figure}[H]
            \includegraphics[width=\textwidth]{Images/SEVESO.jpg}
            \centering
            \caption{Les sites Seveso en France : Officiel Prévention 2020}
        \end{figure}
        
        Selon la \gls{DGPR}, la France comptait environ 23 000 installations autorisées en 2023, dont près de 1 300 sites Seveso (seuil haut ou bas) soumis à un suivi renforcé. Le régime \gls{ICPE} est très important pour qualifier le niveau de dangerosité d'un site mais le type de danger est identifié par la rubrique \gls{ICPE}.
        
        \subsubsection{Les rubriques \gls{ICPE} : outils de qualification des risques}
        
        Chaque rubrique précise les critères techniques ou quantitatifs qui déclenchent un régime \gls{ICPE}. Par exemple, la rubrique 2712 concerne les installation d'entreposage, dépollution, démontage ou découpage de véhicules hors d’usage ou de différents moyens de transports hors d'usage, à l'exclusion des installations visées à la rubrique 2719 et le régime applicable à l'installation dépend de la surface du site ou du type de véhicules. 
        
        \begin{figure}[H]
            \includegraphics[width=300]{Images/Rubrique.png}
            \centering
            \caption{Rubrique 2712 et régime \gls{ICPE} : \gls{AIDA} \gls{INERIS} 2025}
        \end{figure}
        
        Les rubriques sont donc directement en relation avec la nature de l’activité, les substances utilisées ou produites et enfin leur quantité, et décident de la nature des éléments à intégrer dans le dossier administratif à constituer. Selon les rubriques concernées, l’exploitant sera soit tenu de produire une étude d’impact environnemental, une étude de dangers, un plan d’opération interne (POI), soit soumis à des exigences techniques spécifiques figurant dans des arrêtés ministériels modèles.\\

        L’existence d’une nomenclature nationale permet en effet d’assurer une classification homogène du parc des installations à l’échelle du territoire national garantissant ainsi un certain degré d’équité et de cohérence dans l’instruction des dossiers. Si cette classification s’appuie sur des critères objectifs, elle laisse cependant la possibilité d’une marge d’interprétation, notamment à l’échelle locale. En effet, la localité peut avoir son mot à dire, tant en raison des réalités territoriales, des caractéristiques du site, de son histoire, ou encore surtout en raison des pratiques des services instructeurs, qui peuvent diverger d’un niveau territorial à un autre, d’une région à une autre par exemple.
        
        \subsubsection{Les arrêtés ministériels : cadre technique et normatif}
        
        Pour chaque rubrique ou catégorie de rubriques, des arrêtés ministériels de prescriptions communes déterminent les règles de fonctionnement des installations.\\

        Ils imposent des obligations précises : systèmes de rétention, moyens de lutte contre l’incendie, auto-contrôle, prévention des nuisances sonores ou olfactives, suivi des émissions, gestion des déchets secondaires, etc. Par exemple, les installations à autorisation pour les rubriques 2710/2712/2718/2790 et 2791 partagent le même enregistrement dont un extrait d’article requérant l’établissement et la mise à jour d’un plan de défense incendie est ci-après (et les parties de conditions exigées) :

        \begin{figure}[H]
            \includegraphics[width=\textwidth]{Images/Arrêté.png}
            \centering
            \caption{Extrait de l'article 5 de l'arrêté du 22/12/23 relatif à la prévention du risque d'incendie au sein des installations soumises à autorisation au titre des rubriques 2710, 2712, 2718, 2790 ou 2791 de la nomenclature \gls{ICPE}: \gls{AIDA} \gls{INERIS} 2025}
        \end{figure}

        La connaissance et la bonne application des arrêtés sont indispensables pour tout exploitant ou bureau d’études chargé du montage d’un dossier \gls{ICPE}. En cas de non-respect, les sanctions peuvent aller de la mise en demeure à la fermeture administrative du site. La remise en conformité d'un site peut se faire pour l'industriel à l'aide d'un bureau d’études spécialisé dans les \gls{ICPE} comme celui dans lequel j'ai fait mon stage.

    \subsection{Le rôle d’un bureau d’études spécialisé dans les \gls{ICPE}}

    Fondé il y a une vingtaine d’années, EVOLUTYS est un bureau d’études spécialisé dans les Installations Classées pour la Protection de l’Environnement (\gls{ICPE}). Après avoir été implanté à Nîmes (30), l’entreprise s’est installée dans la région lyonnaise avant de rejoindre L’Horme (42) en 2022. Bien qu’intervenant principalement sur des entrepôts logistiques, le domaine d'expertise d'EVOLUTYS en \gls{ICPE} est aussi varié qu'il existe de type d'\gls{ICPE}.\\

    EVOLUTYS assiste les entreprises dans :
    \begin{itemize}
        \item L’élaboration des dossiers \gls{ICPE} 
        \item Le suivi en phase d’exploitation (audits de conformité, porter-à-connaissance)
    \end{itemize}

    La procédure de porter-à-connaissance permet de mettre à jour le régime et/ou la rubrique \gls{ICPE} d'une installation en cas de modification de l'activité. 
    
    La réglementation \gls{ICPE} évoluant constamment, les entreprises doivent s’assurer que leurs installations restent conformes. EVOLUTYS réalise des audits approfondis comprenant :
        \begin{itemize}
        \item Un bilan du classement \gls{ICPE} (état des stocks, actualisation de la nomenclature)
        \item La vérification des contrôles périodiques (électricité, incendie, etc.)
        \item L’examen de la conformité par rapport à l’arrêté préfectoral et aux arrêtés ministériels (incluant une visite terrain que j'ai pu faire par exemple dans des entrepôts logistique)
        \item Une synthèse des non-conformités (classées par niveau de gravité) avec des préconisations de mise en conformité
    \end{itemize}

    À mon avis, l’audit de conformité, souvent négligé, pourrait tout aussi bien être généralisé et systématisé en tant qu’exigence annuelle pour l’ensemble des installations classées à enregistrement ou à autorisation, peu importe leur rubrique, afin de reconnaître des manquements potentiels à la réglementation tout en prévenant efficacement des accidents ou catastrophes environnementales et industrielles, ce qui permettrait de garantir la sécurité des personnes, la protection de l’environnement et la pérennité des activités. Pour être déléguer cette compétence au niveau départemental avec la possibilité de faire appel à des entreprises privées pour le faire dans un premier temps serait intéressant.\\

    La réglementation régissant les \gls{ICPE} est un cadre juridique particulièrement prescriptif qui repose sur des obligations techniques, organisationnelles et administratives concernant tout ce qui touche aux activités susceptibles de présenter un risque ou un impact sur l’environnement voire sur la santé publique. Pourtant, il arrive que certains exploitants ne respectent pas ces exigences, faute d’information ou en raison d’un cadre complexe. Plus préoccupant encore, certains industriels tendent à relativiser, voire à négliger ces obligations au nom de considérations économiques ou de rentabilité, mettant ainsi en péril non seulement leur propre activité mais également les territoires sur lesquels ils opèrent.
    
    \subsection{Les \gls{ICPE} comme cadre de régulation pour les installations de recyclage - Conclusion}

    La réglementation des Installations Classées pour la Protection de l’Environnement (\gls{ICPE}), instaurée par la loi du 19 juillet 1976, constitue un pilier essentiel de la politique française de prévention des risques environnementaux et sanitaires. Elle vise à encadrer les activités industrielles et agricoles présentant des nuisances ou des dangers pour la santé humaine, l’environnement ou les biens, à travers un régime juridique structuré et hiérarchisé. Dans le domaine de la gestion des déchets, cette réglementation prend une importance particulière, tant ces activités peuvent générer des pollutions multiples (lixiviats, émissions atmosphériques, nuisances sonores et olfactives) et sont appelées à se développer dans le contexte de la transition vers une économie circulaire.

    Les \gls{ICPE} offrent un cadre normatif évolutif et proportionné, fondé sur la nature des substances traitées, les volumes concernés et les technologies mobilisées. La nomenclature, composée d’environ 200 rubriques, permet de classer chaque installation selon son niveau de risque, et d’appliquer l’un des trois régimes correspondants : déclaration, enregistrement ou autorisation. Cette classification est complétée par des arrêtés ministériels qui fixent les prescriptions techniques et environnementales à respecter.

    Face aux défis environnementaux croissants — notamment la hausse continue de la production mondiale de déchets, la pollution des milieux, et les pressions sur les ressources — le cadre \gls{ICPE} constitue un levier de maîtrise des risques, mais aussi un révélateur des tensions structurelles. D’une part, il sécurise le développement des installations de traitement et de recyclage en imposant des normes strictes. D’autre part, il peut, en raison de sa complexité et de sa lourdeur administrative, freiner certaines dynamiques d’innovation ou de déploiement territorial, en particulier dans les zones à fortes contraintes environnementales ou à vulnérabilité sociale marquée.\\

    Dans ce contexte, la mise en place d’une installation de traitement ou de valorisation des déchets nécessite une maîtrise rigoureuse des démarches réglementaires, et notamment des conditions liées au régime des \gls{ICPE}. L’élaboration d’un dossier \gls{ICPE} constitue une étape centrale du processus administratif, conditionnant la faisabilité juridique et opérationnelle du projet. Elle implique de croiser plusieurs dimensions : analyse des impacts environnementaux, respect des prescriptions techniques, concertation avec les parties prenantes locales, et conformité aux orientations des politiques publiques territoriales.
    \newpage
\sloppy
\section{Élaboration d’un dossier \gls{ICPE} pour une installation de recyclage : méthode et outils}

    La mise en place d’une installation de recyclage, du fait de son activité potentiellement polluante ou dangereuse, est soumise au régime des Installations Classées pour la Protection de l’Environnement (\gls{ICPE}). Dans ce cadre, l’élaboration d’un dossier \gls{ICPE} représente une étape cruciale, à la fois pour garantir la conformité réglementaire du projet, mais aussi pour en assurer l’acceptabilité environnementale, territoriale et sociale.\\

    Le présent dossier, défini au regard du Code de l’environnement et des arrêtés ministériels pris pour chacun des types de déchets pris en compte (plastiques, bois, métaux, etc.), s’appuie sur une série d’analyses et de documents techniques qui lui permettent d’évaluer les risques et nuisances liés au site, de proposer des mesures de réduction ou de compensation, et de démontrer la compatibilité du projet avec son environnement.

    Pour les acteurs du recyclage, cette procédure constitue une contrainte administrative ou plutôt une opportunité : elle représente un problème parce que le cadre réglementaire d’aménagement n’est pas facile à appréhender et l’ingénierie environnementale complexe à maîtriser, mais une chance dans la mesure où elle peut permettre la mise en œuvre de dispositifs plus performants, plus durables et plus respectueux des conditions d’intégration du site dans son environnement.\\

    Cette partie s’attachera à exposer la méthode d’élaboration d’un dossier \gls{ICPE} dans le cadre d’un projet de recyclage, en s’appuyant sur les outils mobilisables (référentiels, modélisations, outils SIG, données réglementaires) et sur les différentes étapes-clés : cadrage initial, choix du régime \gls{ICPE}, évaluation des impacts, prise en compte du contexte territorial, et mesures d’atténuation environnementale.
 
    \subsection{Phase de cadrage : caractérisation du projet et de ses enjeux}
    
    Dans tout projet relevant du régime des \gls{ICPE}, la première étape consiste à établir un échange avec le porteur de projet afin de bien comprendre ses intentions, ses options et les caractéristiques générales de l’installation prévue.\\

    Chez EVOLUTYS, cela débute par la transmission d’un questionnaire détaillé, au format Excel, que le client complète et puis une réunion avec le client. Ce document constitue un outil de cadrage initial essentiel : il permet de collecter toutes les données techniques, organisationnelles et réglementaires nécessaires à la bonne compréhension du projet. On y trouve les données sur l’exploitant, le site, sa nature et sa situation, les rubriques \gls{ICPE} envisagées, les dispositifs de second œuvre (y compris les dispositifs de lutte contre l’incendie), l’organisation du personnel, les flux logistiques (et notamment le trafic poids lourd), les consommations énergétiques, les émissions de polluants atmosphériques, la gestion de l’eau et la gestion des déchets notamment en fin de vie de l’installation et en phase d’exploitation.\\

    Le questionnaire permet également de préfigurer les études complémentaires à intégrer au dossier (étude de dangers, étude faune-flore, etc.), tout en rassemblant les plans techniques et les documents supports transmis par le client.

    Cette phase de cadrage représente une étape fondamentale dans l’élaboration du dossier \gls{ICPE}, en permettant la mise en place d'un calendrier et d’identifier la nature du projet. Malheureusement, je n’ai pas pu assister aux réunions ou encore à la réalisation des questionnaires, mais les informations récoltées me furent grande utilité dans la suite de l’élaboration des dossiers.  
    
    \subsection{Prise en compte des exigences réglementaires spécifiques au recyclage}
    
    Comme nous avons pu l’observer, les installations de gestion des déchets, bien qu’indispensables dans le cadre d’une économie circulaire et d’une gestion durable des ressources, ne sont pas dénuées d’effets secondaires potentiellement nuisibles. Qu’il s’agisse de nuisances olfactives, sonores, visuelles, ou encore d’émissions de polluants dans l’air, dans les sols ou les eaux, ces équipements peuvent susciter des préoccupations légitimes, notamment lorsqu’ils sont implantés à proximité de zones habitées. En raison de ces impacts potentiels, les installations de recyclage sont soumises à une réglementation stricte, relevant du régime des \gls{ICPE}.\\

    Toutefois, l’encadrement réglementaire de ces installations n’est pas uniforme. Il comporte un ensemble de spécificités propres à chaque type d’installation, déterminées par plusieurs critères interdépendants. Le premier facteur est la nature des déchets traités, qui peut aller de déchets inertes à des déchets dangereux ou biodégradables, chacun impliquant des contraintes techniques, des risques spécifiques et des exigences réglementaires distinctes. Le second critère réside dans les procédés techniques employés : les technologies de tri mécanique, de compostage, d’incinération ou de méthanisation, par exemple, n’ont pas les mêmes impacts ni les mêmes exigences en matière de surveillance ou de confinement. 
    
        \subsubsection{Des rubriques \gls{ICPE} dédiées aux opérations de tri, transit, regroupement ou traitement}
        
        La nomenclature \gls{ICPE} intègre une série de rubriques spécialement dédiées aux activités de gestion des déchets. Celles-ci sont principalement regroupées dans les familles 27XX de la nomenclature. Parmi les plus courantes pour les projets de gestion des déchets, on peut citer :
        
        \begin{itemize}
            \item Rubriques 2710 à 2720 et 2760 — Opérations de gestion/transit/stockage
            \item Rubriques 2770 à 2771 — Traitement thermique
            \item Rubriques 2780 à 2783 — Traitements biologiques
            \item  Rubriques 2790 à 2798 — Traitement de déchets spécifiques
        \end{itemize}

        Le seuil de tonnage annuel, les déchets traités ou les procédés mis en œuvre déterminent le régime applicable : déclaration, enregistrement ou autorisation. Par exemple, le régime pour une installation de transit, regroupement, tri ou préparation en vue de réutilisation de déchets non dangereux non inertes dépend du volume de déchets susceptibles d'être présents dans l'installation : déclaration entre 100 m³ et 1000 m³, enregistrement à partir de 1000 m³ (Rubrique 2716).
        
        \subsubsection{Des prescriptions techniques spécifiques et renforcées}
       
        Les installations de recyclage sont tenues de respecter des prescriptions techniques strictes définies par arrêtés ministériels de prescriptions générales. 
        Ces arrêtés fixent des obligations précises par exemple en matière de : 

        \begin{itemize}
            \item Stockage sécurisé des déchets : couverture des bennes, protection contre l’infiltration des eaux pluviales, séparation des flux dangereux ;
            \item Maîtrise des pollutions accidentelles : dispositifs de rétention, plans d’eau à proximité, zones d’imperméabilisation ;
            \item Contrôle des émissions : limitation des envols de poussières, gestion des odeurs, niveau sonore réglementé ;
            \item  Contrôle des émissions : limitation des envols de poussières, gestion des odeurs, niveau sonore réglementé ;
            \item Prévention des incendies, avec des exigences renforcées depuis plusieurs incendies médiatisés
            \item Des mesures spécifiques de surveillance environnementale (analyses de lixiviats, suivi des eaux de ruissellement, contrôles périodiques) sont souvent exigées, en particulier pour les installations autorisées ou traitant des déchets dangereux [\gls{DGPR}, 2022].
        \end{itemize}

        \subsubsection{Prise en compte des dispositions réglementaires}
        
        Une fois le classement \gls{ICPE} établi lors de la phase préliminaire du projet, il est essentiel de consulter les arrêtés ministériels correspondant aux rubriques \gls{ICPE} concernées. Ces textes réglementaires précisent les prescriptions techniques et environnementales à respecter, et constituent une référence incontournable pour orienter les choix de conception. Cette étape est déterminante, car elle permet de guider efficacement le maître d’ouvrage, le constructeur et l’architecte sur les mesures à intégrer dès la phase de conception. Elle représente ainsi le cœur du rôle du consultant \gls{ICPE}, garant du respect de la réglementation dès l’origine du projet. \\

        Les arrêtés ministériels relatifs aux rubriques \gls{ICPE} sont disponibles sur la plateforme \gls{AIDA}, développé par l’\gls{INERIS} .
        Par exemple, les arrêtés ministériels relatifs aux \gls{ICPE} imposent souvent des distances minimales d’implantation et des mesures de réduction des nuisances (écrans acoustiques, aménagement paysager, horaires de fonctionnement limités). En effet, si mon champs de pommes de terre contenait des casiers (subdivision de la zone à exploiter assurant l'indépendance hydraulique, délimitée par des flancs et un fond) d'une installation de stockage de déchets non dangereux d'une capacité totale supérieure à 25 000 tonnes ou recevant plus de 10 tonnes de déchets par jour, l'arrêté du 15 février 2016 s'appliquerait. \\
        
        Ainsi notamment l'article de 7 de ce présent arrêté qui stipule que \textit{"Afin d'éviter tout usage des terrains périphériques incompatible avec l'installation, les casiers sont situés à une distance minimale de 200 mètres de la limite de propriété du site. Cette distance peut être réduite si les terrains situés entre les limites de propriété et la dite distance de 200 mètres sont rendus inconstructibles par une servitude prise en application de l'article L. 515-12 du code de l'environnement pendant la durée de l'exploitation et de la période de suivi du « casier », ou si l'exploitant a obtenu des garanties équivalentes en termes d'isolement sous forme de contrats ou de conventions pour la même durée."} s'appliquerait.\\

        A l'aide d'extraits cadastrals comme ci-dessous avec l'analyse de l'environnement de mon champs de pommes de terre, on observe que l'installation de casiers serait purement non-réglementaire au regard de l'arrêté du 16 février 2015.\\

        \begin{figure}[H]
            \includegraphics[width=\textwidth]{Images/Cadastre.png}
            \centering
            \caption{Analyse des abords de mon champs de pommes de terre : CRETEUR Tom 2025}
        \end{figure}
        
       La consultation des services de l’État (\gls{DREAL}) et des Services Départementaux d’Incendie et de Secours lors de la construction du dossier permet d’anticiper les éventuelles oppositions et d’adapter le projet à ses contraintes. Cependant, un autre aspect important reste la connaissance des incidences environnementales d'une installation.
     
    \subsection{Analyse des incidences environnementales}
    
   L'analyse des incidences environnementales fait partie du processus d’instruction des projets soumis à la réglementation des \gls{ICPE}. Elle ne doit pas être perçue comme une simple obligation technique mais bien comme l’élément du dispositif visant à vérifier la compatibilité d’un projet avec son environnement physique (géographie, nature des sols, hydrologie), humain (présence d’habitations ou d’équipements sensibles) et réglementaire (plans d’urbanisme, zonages environnementaux, servitudes, etc.). C’est donc aussi un outil d’aide à la décision pour les autorités mais également une possibilité de garantir que la mise en œuvre d’un projet de développement d’activité potentiellement polluante ou à risque s’inscrit dans un cadre territorial logique et permet le respect du fonctionnement des équilibres locaux.\\

    Les projets d'équipements dédiés à la gestion des déchets sont le lieu d’une analyse territoriale particulière ; tout en inscrivant l’activité dans une dynamique vertueuse de valorisation des déchets, ils doivent conjuguer plusieurs impératifs souvent contradictoires car, d’une part leur efficacité logistique suppose une proximité avec les flux de déchets à traiter : déchetterie, tri, valorisation, et d’autre part son implantation doit se situer suffisamment loin des zones sensibles pour respecter, en cas de situation accidentelle, la sécurité des personnes (résidences, écoles, centres publics, etc.), en raison des nuisances liées à ces installations, notamment celle des types de déchets qui seront traités dans l’unité.
        
        \subsubsection{Compatibilité avec les documents d’urbanisme}
        
        L’un des premiers éléments à examiner est la conformité du projet avec les documents d’urbanisme locaux : Plan Local d’Urbanisme (\gls{PLU}) ou Plan Local d’Urbanisme interommunal(PLUi) et Schéma de Cohérence Territoriale (\gls{SCoT}). Ces documents définissent les zones où les activités industrielles sont autorisées et fixent, le cas échéant, des prescriptions particulières (servitudes, zones protégées, contraintes de hauteur ou d’accès).
        A titre d'exemple, voici la compatibilité de mon champs de pommes de terre avec un article du Plan Local d'Urbanisme de ma commune en fonction de la zone où il est implanté (voir "Localisation de mon champs de pomme de terre").

        \begin{figure}[H]
            \includegraphics[width=370]{Images/Localisation patates.png}
            \centering
            \caption{Localisation de mon champs de pomme de terre : CRETEUR Tom 2025}
        \end{figure}

        \begin{figure}[H]
            \includegraphics[width=390]{Images/Patates.png}
            \centering
            \caption{Compatibilité de mon champs de pomme de terre avec le PLU : CRETEUR Tom 2025}
        \end{figure}
        
        
        Enfin, la création d’une \gls{ICPE} dans un secteur inadapté pour ce type d’activités tel qu’un espace à vocation résidentielle, agricole ou à forte sensibilité écologique compromet sérieusement la faisabilité du projet. Implanter une telle installation dans un contexte territorial inadapté peut, en effet, conduire soit à un refus pur et simple d’autorisation de la part de l’administration soit à l’imposition de prescriptions techniques et d’aménagements extrêmement contraignants destinés à limiter ses impacts sur l’environnement et le cadre de vie local.\\

        Ainsi, il est nécessaire d’anticiper en amont les contraintes réglementaires pesant sur un site d’implantation. Dans ce sens, la plateforme Géoportail de l’Urbanisme constitue un outil très utile. En effet, elle permet de consulter les documents d’urbanisme officiels applicables à une parcelle donnée : \gls{PLU}, zones inconstructibles, et autres prescriptions. Dans le cas où l’information n’est pas disponible sur cette plateforme, se référer aux sites internet des mairies concernées dans les rubriques urbanisme peut également fournir les documents nécessaires pour évaluer la compatibilité du projet avec la réglementation locale.\\

        Un autre point important à explorer dans le cadre de l’analyse comprises les éventuelles servitudes d’utilité publique (\gls{SUP}). Ainsi, selon le \gls{CEREMA}, une \gls{SUP} est une réglementation définie comme la « limitation administrative au droit de propriété, instituée par la puissance publique dans un but d’utilité publique ». Ces restrictions se situent dans une variété des domaines, y compris la sécurité, la protection de l’environnement, la conservation du patrimoine...\\

        Ces servitudes se répartissent en quatre grandes catégories :
        
        \begin{itemize}
            \item Conservation du patrimoine naturel, culturel et sportif ;
            \item Protection des ressources et équipements (réseaux d’énergie, canalisations, télécommunications, etc.) ;
            \item Défense nationale ;
            \item Salubrité et la sécurité publique.
        \end{itemize}

        Lorsqu’un projet est localisé dans le périmètre d’une SUP, il peut être soumis à des prescriptions spécifiques, qu’il est impératif d’identifier en amont afin d’anticiper d’éventuelles contraintes techniques ou administratives. D'autres contraintes techniques peuvent apparaître en cas de risques naturels et technologiques.

        \subsubsection{Évaluation des risques naturels et technologiques liés à l’environnement du site projeté}

        On définit le risque comme exposition d’une cible (salarié, entreprise, environnement) à un danger. Le risque est caractérisé par la combinaison de la probabilité d’occurrence d’un événement redouté (accident) et de la gravité de ses conséquences. Il est essentiel de prendre en compte les risques naturels et technologiques susceptibles d’affecter l’emprise du site lors de l'élaboration d'un dossier \gls{ICPE}. Pour cela, la plateforme Géorisques, développée par le Bureau de Recherches Géologiques et Minières (\gls{BRGM}), constitue un outil de référence. Elle permet d’obtenir un rapport de risque à une certaine adresse pour 10 risques référencés (exemple ci-dessous pour mon champs de pommes de terre).

        \begin{figure}[H]
            \includegraphics[width=\textwidth]{Images/Géorisques.png}
            \centering
            \caption{Rapport de risques pour mon champs de pommes de terre : CRETEUR Tom 2025}
        \end{figure}
 
        L’analyse de ces risques permet, le cas échéant, d’adapter ou annuler le projet en conséquence, afin de garantir sa conformité réglementaire et sa résilience face aux aléas identifiés dans l’environnement immédiat du site. En plus des risques naturels et téchnologiques, les installations \gls{ICPE} sont également confrontées à leurs impacts environnementaux pour leur installation. 
        
        \subsubsection{Impacts environnementaux}
        
        L'identification des enjeux environnementaux potentiels susceptibles d’interagir avec un projet \gls{ICPE} se fait via Géoportail. Le tableau ci-dessous présente la liste  des types d’espaces protégés disponibles sur Géoportail, qui sont pris en compte dans le cadre des dossiers \gls{ICPE}.

        \begin{longtable}{|>{\raggedright\arraybackslash}p{4.5cm}|>{\raggedright\arraybackslash}p{10.5cm}|}
        \caption{Types de données recensées sur Géoportail permettant l’analyse des sensibilités environnementales des dossiers \gls{ICPE}} \\
        \hline
        \textbf{Type d’espace protégé} & \textbf{Description} \\
        \hline
        \endfirsthead

        \hline
        \textbf{Type d’espace protégé} & \textbf{Description} \\
        \hline
        \endhead

        \gls{ZNIEFF} (Zones Naturelles d’Intérêt Écologique, Faunistique et Floristique) & 
        Représente les secteurs du territoire ayant un grand intérêt écologique abritant une biodiversité patrimoniale importante.

        Il en existe 2 types :
        \begin{itemize}
        \item Type I: représente des « espaces homogènes écologiquement », définis par la présence d’espèces, d’associations d’espèces ou d’habitats rares, remarquables ou caractéristiques du patrimoine naturel régional.
        \item Type II: recense des « espaces qui intègrent des ensembles naturels fonctionnels et paysagers, possédant une cohésion élevée et plus riches que les milieux alentour » (INPN, s.d.).
        \end{itemize}
        \\
        \hline

        Natura 2000 & Les sites Natura 2000 ont vocation à protéger certains habitats et certaines espèces dites « d’intérêt communautaire » en assurant un équilibre entre protection et activités humaines.\\
        \hline

        Parc national & Il existe 11 parcs nationaux sur le territoire français, tous rattachés à l’OFB (Office Français de la Biodiversité). Ce sont des territoires d’exception, terrestres ou maritimes, dont les cœurs font l’objet d’une réglementation spécifique visant à préserver les richesses.\\
        \hline

        Parc naturel régional (PNR) & Les PNR ont pour objectif la protection et la mise en valeur du patrimoine naturel, culturel et humain de territoires remarquables pour leurs paysages, milieux naturels ou patrimoine culturel mais fragiles.\\
        \hline

        Parc naturel marin (PNM) & 
        Les PNM permettent de protéger des espaces marins de très grandes étendues. Les objectifs des PNM sont de protéger, développer la connaissance du patrimoine marin et promouvoir le développement durable des activités professionnelles et de loisirs liées à la mer.\\
        \hline

        Réserve Naturelle Nationale ou Régionale & 
        Les réserves naturelles, qu’elles soient nationales ou régionales, ont pour vocation, entre autres, la préservation de la biodiversité des zones remarquables, la reconstitution de populations animales ou végétales, la prévention des perturbations et formations géologiques ou encore la préservation d’étapes sur les grandes voies de migration de la faune sauvage.\\
        \hline

        \end{longtable}
        
        Ci-dessous se trouve un extrait d'une étude d'incidence de l'environnement naturel et plus précisément ici du réseau Natura 2000 pour mon quartier de résidence : \\ 

        \textbf{Réseau Natura 2000} \\

        L'objectif est d’identifier un réseau représentatif et cohérent d’espaces permettant d’éviter la disparition de milieux et d'espèces protégées.
        Les inventaires dits "Natura 2000" correspondent à des territoires comportant des habitats naturels d’intérêt communautaire et/ou des espèces d’intérêt communautaire. Les "habitats naturels" (en général définis par des groupements végétaux) et les espèces d’intérêt communautaire présents en France font l’objet de deux arrêtés du Ministre chargé de l’environnement en date du 16 novembre 2001 (JO du 29/01/2002).

        Dans ces périmètres, il convient de vérifier que tout aménagement ne porte pas atteinte à ces habitats ou espèces.
        Le réseau Natura 2000 est constitué :
        
        \begin{itemize}
            \item des Zones de Protection Spéciale (directive Oiseaux)
            \item des Zones Spéciales de Conservation (directive Habitats)
        \end{itemize}

        Les deux zones sont a priori indépendantes l’une de l’autre, c'est-à-dire qu’elles font l’objet de procédures de désignation spécifiques (même si le périmètre est identique).\\

        DIRECTIVE HABITATS\\

        La directive n°92-43 du 21 mai 1992, dite directive "Habitats", vise à "contribuer à assurer la biodiversité par la conservation des habitats naturels ainsi que de la faune et de la flore sauvages sur le territoire européen des Etats membres".

        Les Sites d'Importance Communautaire (SIC) sont les sites sélectionnés, sur la base des propositions des États membres, par la Commission européenne pour intégrer le réseau Natura 2000 en application de la directive "Habitats". La liste de ces sites est arrêtée par la Commission Européenne de façon globale pour chaque région biogéographique. Ces sites sont ensuite désignés en ZSC par arrêtés ministériels.

        La ZSC la plus proche du site est la suivante : Identifiant : FR8201762 / Nom : Vallée de l’Ondenon, contreforts nord du Pilat / Distance : 1,5 km au Sud-Est\\

        DIRECTIVE OISEAUX\\

        La directive n°79-409 du 6 avril 1979, dite directive "Oiseaux", relative à la conservation des oiseaux sauvages, s’applique à tous les Etats membres de l’Union Européenne. Elle préconise de prendre "toutes les mesures nécessaires pour préserver, maintenir ou rétablir une diversité et une superficie suffisante d’habitats pour toutes les espèces d’oiseaux vivant naturellement à l’état sauvage sur le territoire européen".

        Cette directive prévoit la création de Zones de Protection Spéciales (ZPS) afin d’assurer la conservation d’espèces d’oiseaux jugées d’intérêt communautaire.
         
         La ZPS la plus proche du site est la suivante : Identifiant : FR8212014 / Nom : Gorges de la Loire / Distance : 8,5 km au Nord-Ouest\\

        \begin{figure}[H]
            \includegraphics[width=\textwidth]{Images/Natura 2000.png}
            \centering
            \caption{Sites Natura 2000 à proximité de mon quartier : CRETEUR Tom 2025}
        \end{figure}

        \textbf{La zone d’étude n’est pas située dans le périmètre de protection d’une ZPS, d’un SIC ou d’une ZSC}\\

        Conformément à l’article R.122-2 du Code de l’environnement, une évaluation environnementale peut être requise, soit sous la forme d’un examen au cas par cas, soit par la réalisation d’une étude d’impact complète, lorsque des enjeux environnementaux significatifs sont identifiés.
        
        En complément de l’analyse cartographique, il est possible de solliciter une étude faune-flore auprès d’un bureau d’études spécialisé. Cette expertise vise à détecter la présence d’espèces protégées ou d’habitats naturels sensibles sur le site concerné. Lorsque cette sensibilité est absente, l’impact environnemental du projet reste limité. En revanche, si des enjeux sont mis en évidence, des mesures \gls{ERC} (Éviter, Réduire, Compenser) peuvent être exigées par l’autorité environnementale compétente ou le projet tout simplement refusé sur cette zone.\\

        Dans la pratique, certains projets ont toutefois été implantés sur des zones écologiquement sensibles sans que des mesures de préservation ou de compensation aient été mises en œuvre, soulevant des interrogations personnelles quant à l’efficacité réelle des dispositifs de protection de la biodiversité. Ces interrogations personnelles subsistent au niveau de la rationnalité territoriale de certains projets.

        \subsubsection{Facilité d’accès logistique et cohérence d’implantation territoriale}
       
        Enfin, l’analyse d’intégration territoriale ne peut faire l’économie d’une évaluation de la connectivité logistique du site. Une installation de recyclage doit être facilement accessible par la route (et si possible par le rail ou des voies navigables), à proximité des flux de déchets produits ou collectés. Ce critère vise à réduire les distances de transport, les émissions de gaz à effet de serre associées et les coûts logistiques.\\
        
        L’article L541-1 du Code de l’environnement, réformé par la loi \gls{AGEC}, incite à une  planification territoriale cohérente de la gestion des déchets. Ainsi, les SRADDET (Schémas Régionaux d'Aménagement, de Développement Durable et d’Égalité des Territoires) intègrent désormais des objectifs d’implantation raisonnée des infrastructures de gestion pour limiter les déséquilibres régionaux.\\
        
        Les SRADDET contiennent, depuis la Loi Notre de 2015, les Plans régionaux de prévention et de gestion des déchets (ou PRPGD) lequels sont des documents règlementaires de planification d'amélioration de la gestion des déchets, tenant compte à la fois des objectifs de la loi, et des particularités régionales.

        \subsubsection{La gestion de l’eau dans les projets \gls{ICPE} : entre contraintes réglementaires et efficacité territoriale}

        La gestion de l’eau constitue un enjeu central dans le développement des projets industriels soumis à la réglementation des Installations Classées pour la Protection de l’Environnement (\gls{ICPE}). Parmi les thématiques les plus sensibles, la maîtrise des eaux pluviales s’impose comme un élément structurant, à la croisée des obligations réglementaires, des attentes locales et des réalités hydrauliques du territoire.\\

        Les projets doivent s’inscrire dans le cadre fixé par le Schéma Directeur d’Aménagement et de Gestion des Eaux (\gls{SDAGE}), document de planification élaboré à l’échelle de chaque bassin hydrographique, ainsi que, le cas échéant, dans les orientations plus locales du Schéma d’Aménagement et de Gestion des Eaux (\gls{SAGE}). Ces documents, encadrés par la Directive Cadre sur l’Eau (2000/60/CE), ont pour ambition de maintenir, voire de restaurer, le bon état écologique des masses d’eau à l’échelle nationale et européenne. En pratique toutefois, leur application dans les projets \gls{ICPE} s’avère inégale à mon sens.\\

        En effet, les arrêtés ministériels \gls{ICPE} imposent une distinction entre eaux claires (toiture) et eaux potentiellement polluées (voirie, quais, parkings). Ces dernières doivent impérativement être traitées par un séparateur d’hydrocarbures avant tout rejet, tandis que les premières peuvent être évacuées vers le réseau, sous réserve d’une convention avec la collectivité ou le gestionnaire de la zone d’activité. Dans le cas de l'eau, cette approche conformiste réduit les capacités de développement de solutions plus durables fondées sur la nature (noues végétalisées, toitures végétales, zones tampons) que j'ai faiblement vu lors de mon stage.
        
    \subsection{Élaboration d’un dossier \gls{ICPE} pour une installation de recyclage : Conclusion}  
    
    La création d'un dossier \gls{ICPE} est une étape cruciale pour mettre en place une installation de recyclage. Elle détermine non seulement la faisabilité réglementaire du projet, mais aussi son intégration dans le territoire concerné. Ce processus a pour but de prouver que l'installation respecte les exigences du Code de l'environnement, tout en anticipant les impacts environnementaux, techniques et sociaux du projet.\\

    Tout commence par une phase de cadrage où les porteurs de projet remplissent un questionnaire technique. Ce questionnaire permet de rassembler des informations sur le site, les rubriques \gls{ICPE} concernées, les flux logistiques, les dispositifs de sécurité, la gestion de l'eau, les émissions atmosphériques, et même l'organisation du personnel. Ce cadrage sert ensuite de base pour les études environnementales et techniques nécessaires à la constitution du dossier.\\

    Prendre en compte les exigences réglementaires spécifiques au recyclage est vraiment crucial. Ces exigences varient en fonction de la nature des déchets que l'on traite, qu'ils soient inertes, dangereux ou biodégradables, ainsi que des techniques utilisées, comme le tri, le compostage ou l'incinération. La nomenclature \gls{ICPE} regroupe les différentes rubriques concernées, notamment dans la série des 27XX, où chacune définit un niveau de danger et un régime procédural approprié, que ce soit une déclaration, un enregistrement ou une autorisation.\\

    Le dossier inclut aussi une analyse des impacts environnementaux, avec des sections consacrées à la compatibilité avec les documents d’urbanisme, aux risques naturels et technologiques, aux effets sur l’environnement (comme les pollutions et nuisances), à la logistique, à la gestion des eaux pluviales, ainsi qu'aux mesures de compensation ou d'atténuation.\\

    Enfin, la réglementation impose une rigueur formelle et technique qui requiert souvent l'intervention de compétences variées, généralement confiées à un bureau d’études spécialisé. Tout ce processus s’inscrit dans une démarche de concertation, de transparence et d’intégration environnementale. Cependant, il met aussi en lumière une certaine complexité qui peut représenter un frein pour les projets innovants ou ceux portés par de petits acteurs.\\

    A quelle répartition du territoire français sont soumis ces équipements ? Quelles populations sont tributaires de leur nuisances, qui bénéficieront à l’inverse des retombées économiques et symboliques de ceux-ci ? Cette question peut-elle réellement soutenir une relocalisation industrielle conçue à la fois équitable et durable ? Autant de questions qui méritent d’être soulevées et qui font l’objet de cette prochaine partie de travail.

    \newpage
\sloppy
\section{Enjeux territoriaux et environnementaux de l’implantation des installations de gestion de déchets}

À l’heure où la transition écologique devient un impératif national et européen, le développement d’infrastructures de recyclage s’impose comme une priorité pour accompagner la mutation vers une économie circulaire. En France, la loi relative à la lutte contre le gaspillage et à l’économie circulaire (loi \gls{AGEC} du 10 février 2020) renforce cette dynamique, en fixant des objectifs ambitieux de réduction des déchets, de valorisation matière et de limitation du recours à l’enfouissement et à l’incinération.\\

Dans ce contexte, les installations de tri, de préparation ou de traitement des déchets se multiplient sur le territoire. Mais leur implantation ne va pas sans poser d'importantes questions territoriales et environnementales. En effet, ces infrastructures sont parfois sources de conflits d’usage, de saturation foncière ou de nuisances perçues par les riverains. Leur implantation doit donc répondre à une logique d’optimisation territoriale : proximité des gisements de déchets, accessibilité logistique, compatibilité avec les documents d’urbanisme, tout en assurant la préservation des milieux naturels et la réduction de l’empreinte écologique.\\

Dès lors, le choix de l’implantation d’une installation de recyclage ne s’entend pas comme devant seulement être guidé par les préoccupations techniques de gestion qui sont les siennes. Sa mise en œuvre doit s’appréhender dans une logique intégrée, recoupant un travail de planification spatiale, de prise en compte de l’acceptabilité sociale et d’évaluation environnementale, qui met en lumière des tensions toujours plus prégnantes entre objectifs de réindustrialisation dorénavant dits « verts », et considérations liées aux aménagements du territoire. La présente partie se propose de faire le tour de ces enjeux, par le canal de trois axes : la typologie des installations de gestion des déchets afin de contextualiser les installations de gestion des déchets, la répartition inégale des installations de gestion des déchets et les bénéfices associés aux projets \gls{ICPE} de gestion des déchets.

    \subsection{Typologie des installations de gestion des déchets}

    L’Hexagone est dotée d’un maillage d’installations spécialisées organisé selon la nature des déchets pris en charge (ménagers, industriels, dangereux, inertes, etc.) et les modes de traitement qui leur sont appliqués (recyclage, incinération, stockage, valorisation énergétique, compostage, etc.). L’organisation territoriale qui en résulte est conçue pour garantir une prise en charge d’efficacité compatible avec les contraintes environnementales et se rapprochant des territoires de production de déchets.\\

    Aussi est important de bien connaître les principales typologies de ces installations pour déceler leurs caractéristiques, avantages et inconvénients. En effet, chaque type de site a de propres spécifications techniques et environnementales qui vont influer sur la performance de son fonctionnement ainsi que sur son intégration dans le territoire, sa concertation avec les citoyens ou son acceptabilité sociale. L’exploration de ces installations apparaît dès lors comme un chemin d’accès à la compréhension des enjeux qui sont liés à la planification et à l’implantation des infrastructures de traitement des déchets en France.

    \begin{longtable}{|>{\raggedright\arraybackslash}p{4.5cm}|>{\raggedright\arraybackslash}p{10.5cm}|}
        \caption{Typologie des installations de gestion des déchets} \\
        \hline
        \textbf{Type d’installations de gestion des déchets} & \textbf{Description} \\
        \hline
        \endfirsthead

        \hline
        \textbf{Type d’installations de gestion des déchets} & \textbf{Description} \\
        \hline
        \endhead

        Les installations de collecte et de tri & 
        La première étape dans la gestion des déchets consiste en leur collecte, qui peut s’effectuer par divers moyens tels que le porte-à-porte ou la mise en dépôt dans des points d’apport volontaire. Une fois collectés, ces déchets sont transportés vers des centres de tri. Ces centres, assurent la séparation des matériaux recyclables — papier, plastique, métal, verre — des flux résiduels, afin de préparer leur traitement ultérieur.\\
        \hline

        Les installations de traitement biologique & Les déchets organiques, lorsqu'ils sont triés et non souillés, sont dirigés vers des installations de traitement biologique. Le compostage, qu’il soit réalisé à l’échelle industrielle ou locale, permet de transformer ces matières en compost, utilisé comme amendement du sol. Un autre procédé est la méthanisation qui à l'aide de archées anaérobies méthanogènes produit du biogaz, tout en générant un résidu nommé digestat, qui peut être réutilisé dans différents contextes.\\
        \hline

        Les installations de valorisation énergétique & Les usines de valorisation énergétique sont des équipements dont le rôle principal est de brûler les déchets résiduels qui n'ont pas pu être recyclés. En les incinérant, ces installations récupèrent la chaleur ou l'électricité qui, à leur tour, transforment ce flux indésirable en véritable source d'énergie.\\
        \hline

        Les installations de stockage des déchets non dangereux & Pour les déchets qui échappent au tri ou au recyclage ou encore à la valorisation, des sites de stockage sont construit. Ces centres reçoivent uniquement des déchets non dangereux et sont bâtis pour empêcher les polluants de se répandre dans l'environnement. Ils disposent en général de collecteurs pour les lixiviats et, si besoin, de systèmes de capture pour les gaz qui s'échappent de la décomposition. \\
        \hline

        Les installations spécialisées pour les déchets dangereux et spécifiques & 
        Certaines catégories de déchets exigent néanmoins des traitements sur mesure. Les résidus dangereux-pigments chimiques ou déchets médicaux par exemple-passent par des procédures physico-chimique, par des incinérateurs à très haute température ou par des décharges construites spécialement pour eux. De la même façon, des installations distinctes sont réservées aux appareils électriques, aux voitures hors d'usage ou encore aux gravats du secteur de la construction.\\
        \hline

        \end{longtable}
        
        Chacune de ces catégories d’installations a ses avantages tant au plan opérationnel qu’environnemental, mais aussi ses limites, qui doivent être intégrées dans toute démarche de planification territoriale et d’évaluation des politiques publiques. Il est aussi nécessaire de procéder à une analyse spécifique afin de déterminer leur place et leur rôle.

            \subsubsection{Avantages et Inconvénients des différentes installations de gestion des déchets}

            Les installations de collecte et de tri occupent une place stratégique dans la chaîne de traitement des déchets, car elles conditionnent la qualité des flux orientés vers les filières de valorisation ou d’élimination. Elles présentent l’avantage de permettre une séparation efficace des matériaux recyclables, favorisant leur retour dans le cycle de production et contribuant aux objectifs de transition vers une économie circulaire. Outre cela, elles contribuent à la structuration d’emplois de proximité et permettent une optimisation logistique des flux de déchets. Cependant, leur efficacité dépend d’une manière assez prépondérante du comportement des usagers et de la qualité du tri à la source. Elle peut aussi entraîner des nuisances, notamment sonores ou de circulation.\\

            Les nstallations de traitement biologique, à savoir les unités de compostage ou de méthanisation, prennent en compte la fraction organique des déchets pour favoriser la production de compost ou d’énergie sous forme de biogaz. Ce type de traitement permet non seulement de réduire le volume de déchets résiduels, mais également de fournir des ressources alternatives s’inscrivant de fait dans une logique de durabilité par la réduction des émissions de gaz à effet de serre. Nécessitant en amont un tri rigoureux, ces installations peuvent être sources de nuisances olfactives, la rentabilité économique dépendant de la régularité et de la qualité des déchets entrants, ainsi que de la viabilité des débouchés pour les produits valorisés.\\

            Les installations de valorisation énergétique, généralement combinées à l’incinération avec récupération d’énergie, permettent de traiter les déchets résiduels non recyclables tout en fournissant de l’électricité et/ou de la chaleur. Elles ont le mérite d’effectuer une réduction importante du volume de déchets tout en apportant une contribution énergétique non négligeable au réseau local. Ce sont des équipements qui atteignent un bon rendement énergétique, sous réserve que la dépollution des fumées soit opérée de façon satisfaisante. Toutefois, les investissements tant lors de la construction que lors des phases d’exploitation se révèlent souvent très élevés. De plus, la présence de telles installations suscite le plus souvent des controverses en raison des risques sanitaires potentiels associés au réseau de chaleur. Enfin, la présence de ces équipements peut avoir un effet d’inertie sur le long terme en freinant la mise en œuvre des efforts de prévention et de réduction à la source en direction des collectivités locales.\\

            Les installations de stockage des déchets non dangereux aussi désignées comme « centres d’enfouissement »; ces équipements sont utilisés en dernier ressort pour les déchets ultimes, autrement dit, ceux pour lesquels il n’existe pas de mode de valorisation. En effet, ces sites permettent d’encapsuler les déchets dans des conditions préalablement spécifiques dans la réglementation, en recourant, entre autres, aux dispositifs facilitant le captage du biogaz et le traitement des lixiviats. Toutefois, ces installations provoquent des emprises terrains non négligeables, affecte le paysage et, potentiellement, toute sa circonscription par la gêne qu’elle provoque (odeurs, envols, infiltration du sol) qui rend la solution mal acceptée sociétalement ; qui plus est, elle n’est pas en cohérence avec l’esprit de réduction et de valorisation des déchets.\\

            Les dispositifs spécialisés dans la gestion des déchets dangereux ou spécifiques (déchets d’activité de soins à risque infectieux, déchets industriels spéciaux…) sont essentiels à la maîtrise des risques sanitaires et environnementaux. Leur existence permet, grâce à des modes de confinement, de neutralisation ou de destruction adaptés, d’éviter la dissémination dans la nature de substances particulièrement toxiques, émettant ainsi des émanations des plus nocives. Pour autant, leur coût de mise en œuvre ainsi que de fonctionnement et d’exploitation est élevé. Par ailleurs, les craintes liées au danger qu’elles génèrent et notamment les effets de la méthode d’implantation marquent une réticence fréquente de la part des riverains concernés.\\

            Ainsi, chacune des catégories de installations a des apports significatifs mais aussi des limites, tant au plan environnemental que social. Leur pertinence ne peut ainsi être mesurée qu’au regard d’une réflexion intégrant les logiques de répartition territoriale respectant les filières de production de déchets.
    
    \subsection{Une répartition inégale des installations de gestion des déchets}
    
    La gestion des déchets et leur traitement constituent aujourd’hui un enjeu majeur tant en matière de politique environnementale que de planification territoriale. En France, les Installations Classées pour la Protection de l’Environnement (\gls{ICPE}) sont au cœur de ce dispositif, encadrant juridiquement les activités susceptibles de générer des nuisances ou des risques pour la santé et l’environnement. Or par exemple, la cartographie des installations d'incinération des déchets ménagers révèle une distribution assez hétérogène. Cette disparité géographique, loin d’être le fruit du hasard, résulte d’une convergence de logiques historiques, économiques, sociales et réglementaires, mais aussi de la façon dont les décisions sont prises au niveau local.\\


    \begin{figure}[H]
        \includegraphics[width=\textwidth]{Images/Incinération.jpg}
        \centering
        \caption{Cartographie des UIOM en France : ITOM 2014}
    \end{figure}

        \subsubsection{Une répartition inégale liée aux dynamiques historiques et socio-économiques}
        
        Dès l'Antiquité, Sun Tzu, général chinois du VIᵉ siècle av. J.-C dans L'Art de la guerre disait au sujet des lieux de bataille : \textit{"Sur la surface de la Terre, tous les lieux ne sont pas équivalents ;"}. Quelques 2500 ans plus tard, l'implantation des installations de gestion des déchets fait l'objet du même constat.\\

        Certaines régions de France, les Hauts-de-France, le Grand Est et certaines zones de la vallée du Rhône concentrent encore aujourd’hui un grand nombre d’installations de gestion des déchets. Ce phénomène géographique n’est pas le fruit du hasard, mais bien l’effet d’une histoire industrielle qui a durablement déterminé leur configuration économique, spatiale et sociale.\\

        Historiquement, ces régions ont développé une économie principalement dans l’industrie lourde (sidérurgie, chimie, textile, métallurgie, etc.) leur laissant un certain héritage infrastructurel. On y trouve souvent à disposition des friches industrielles ou des zones d’activités déjà aménagées, propices à l’implantation d’ICPE, telles que des centres de tri, des incinérateurs ou des plateformes de valorisation. C’est ce que l’on appelle l’inertie territoriale : les choix contemporains d’implantations reposent encore en partie sur des structures héritées du passé.\\

        \begin{figure}[H]
        \includegraphics[width=380]{Images/Déchets dangereux.png}
        \centering
        \caption{Localisation des installations de traitement des déchets dangereux régionaux en 2022 : Observatoire Régional des Déchets de Provences-Alpes-Côte d'Azur 2023}
        \end{figure}

        Ces territoires ayant déjà accueilli des activités industrielles ont tendance à poursuivre cette trajectoire, car ils disposent des équipements techniques, des réseaux de transport (rail, route, fleuves), d’un foncier compatible avec les contraintes réglementaires, ainsi que d’un écosystème économique habitué à cohabiter avec ce type d’infrastructure.\\

        Inversement, les secteurs très résidentiels ou récemment proches des centres urbains denses se révèlent moins acceptables socialement pour les installations de gestion des déchets. En effet, alors qu’ils sont en rapports de proximité avec ce type de sites, les habitants des lieux voisins retiennent de leurs nuisances possibles ces emprises (bruits, odeurs, circulation des camions, menaces environnementales ; sanitaires). Ce facteur contribue à rendre difficile l’implantation de nouvelles installations, dans les zones urbanisées très liées à la densité de population et à l’importance de déchets produits.\\

        La distribution territoriale des installations de gestion des déchets résulte d’un double mécanisme spécial : d’un côté, un espace à vocation industrielle qui a su tirer profit de ses atouts de base pour récupérer ces implantations ; de l’autre, des terrains plus résidentiels ou urbanisés où se manifestent les contraintes sociales, politiques, foncières qui limitent nettement leur implantation. Au-delà de l’impact sur la répartition territoriale des installations de gestion de déchets qu’impliquent ces disparités, ce constat interroge les équilibres à trouver entre efficacité logistique, équité territoriale, acceptabilité sociale tant la question de la gestion durable des déchets est affaire de politique stratégique au niveau du pays.\\

        L'histoire socio-économique de certaines régions aide à expliquer la répartition inégale des installations de gestion des déchets, mais qu'en est-il des nouveaux développements ? Comment pouvons-nous justifier le placement inégal de nouvelles installations ?
        
        \subsubsection{L’instruction des dossiers \gls{ICPE} : un processus dépendant des acteurs et de leur territoire}

        La procédure d'enregistrement ou d'autorisation \gls{ICPE} est rigoureusement encadrée mais elle est également soumise à une forte dimension humaine et territoriale. En théorie, l’instruction du dossier suit une logique administrative claire : le porteur de projet dépose un dossier technique, examiné par la \gls{DREAL} (Direction Régionale de l’Environnement, de l’Aménagement et du Logement), qui formule un avis à destination du Préfet de département, seul décisionnaire final. Le Préfet, représentant de l’État déconcentré, agit ici au nom du Premier ministre.\\

        Cependant, dans la réalité observée, les pratiques administratives varient d’un département à l’autre. Ces variations s’expliquent par plusieurs facteurs : les ressources humaines de la \gls{DREAL}, la culture de travail locale, la sensibilité des services préfectoraux aux questions environnementales, ou encore le contexte politique. En effet, les décisions prises dans le cadre des \gls{ICPE} ne sont pas neutres. Elles sont souvent influencées par les idéologies politiques et les arbitrages territoriaux, notamment entre développement économique et protection de l’environnement.\\

        Dans ce contexte, la réussite d’un dossier \gls{ICPE} repose autant sur la conformité réglementaire que sur la capacité d’anticiper les attentes spécifiques du territoire. Il ne suffit plus de répondre aux normes techniques ; il faut aussi construire un projet "acceptable" pour les acteurs locaux. Cela implique une concertation proactive avec la \gls{DREAL}, mais aussi avec d’autres services publics, comme les SDIS (services d’incendie et de secours), les agences de santé (ARS), les collectivités locales, voire les associations de riverains.\\

        Cette stratégie d’ancrage territorial est d’autant plus essentielle que certaines régions apparaissent plus ouvertes que d’autres à l’installation de projets industriels à fort impact. Dans les zones en reconversion économique ou à fort chômage, les projets \gls{ICPE} peuvent être perçus comme des leviers puissants de développement territorial. Mais cette dynamique soulève aussi la question des inégalités environnementales : en concentrant les risques dans certaines zones, on crée des territoires plus exposés aux nuisances, et donc potentiellement moins favorisés à long terme.
        
        \subsubsection{Les conséquences sociales de cette répartition inégale}

        L'inégale répartition des centres de tri, des déchetteries, des incinérateurs ou encore les centres d’enfouissement technique a des conséquences directes sur les populations concernées, en termes de qualité de vie, de justice environnementale et d’accès aux services publics essentiels.\\

        L’impact social de cette répartition inéquitable peut se manifester de plusieurs façons selon moi :

        \begin{itemize}
            \item Un exposome inégal des populations selon les territoires. La qualité du milieu est impactée négativement par les insallations de gestion des déchets. Or, ces installations sont majoritairement implantées dans des zones défavorisées.

            \item Une stigmatisation territoriale et perte de valeur foncière. Installée pour des raisons économiques et parfois politiques dans des milieux défavorisés, l’image négative associée aux déchets contribue à une forme de stigmatisation symbolique de certains quartiers ou communes, dont l’attractivité résidentielle et économique est durablement affectée.

            \item Une inégalités d’accès au service public de gestion des déchets : Dans certains territoires ruraux ou périphériques, les habitants ou les industriels doivent parcourir de longues distances pour accéder à un équipement spécialisé pour traiter leurs déchets. Ces disparités d’accès renforcent l’exclusion territoriale et limitent la participation citoyenne à la transition écologique.

            \item Une concentration des emplois précaires : Les installations de gestion des déchets génèrent des emplois, mais souvent à faible qualification, avec des conditions de travail difficiles (exposition aux nuisances, horaires décalés, contrats courts). Ces postes sont souvent occupés par des populations déjà précaires ou marginalisées (migrants, intérimaires).

            \item Des retombées économiques locales parfois insuffisantes: Dans certains cas, les profits générés par l’activité de gestion sont captés par de grandes entreprises privées, alors que les externalités négatives (pollution, dévaluation foncière) restent à la charge des collectivités d’accueil.
            
        \end{itemize}

        Cependant, lors de mon stage, j'ai pu observer que des efforts sont entrepris pour favoriser une meilleure répartition des infrastructures et un renforcement du maillage territorial, notamment via la réhabilitation de friches industrielles préconisée dans le Plan régional de prévention et de gestion des déchets des Hauts-de-France. Mais de nombreuses questions d'ordre environnemental restent en suspens.

        \subsubsection{Les conséquences environnementales de cette répartition inégale}

         L'inégale répartition des centres de tri, des déchetteries, des incinérateurs ou encore les centres d’enfouissement technique a des conséquences directes sur l'environnement.

        L’impact social de cette répartition inéquitable peut se manifester de plusieurs façons selon moi :

        \begin{longtable}{|>{\raggedright\arraybackslash}p{4.5cm}|>{\raggedright\arraybackslash}p{10.5cm}|}
        \caption{Impacts sociaux de la répartition inéquitable des installations de gestion des déchets} \\
        \hline
        \textbf{Impacts sociaux} & \textbf{Description} \\
        \hline
        \endfirsthead

        \hline
        \textbf{Impacts sociaux} & \textbf{Description} \\
        \hline
        \endhead

        Surconcentration des nuisances environnementales dans certaines zones & 
        Les installations génèrent divers types de pollutions : émissions atmosphériques (oxydes d’azote, particules fines, dioxines), lixiviats contaminant les sols et les nappes phréatiques, odeurs, bruit, etc.
        
        Ainsi, la qualité de l’air et des milieux naturels est significativement altérée à proximité des installations, comme cela a été pu être documenté autour des incinérateurs de Fos-sur-Mer ou des centres d’enfouissement de la vallée du Rhône. Ces pollutions cumulées peuvent avoir des impacts directs sur la biodiversité locale, la fertilité des sols et la salubrité des eaux.\\
        \hline

        Artificialisation et fragmentation des milieux naturels & L’implantation de grandes installations de gestion implique souvent l’artificialisation de sols agricoles ou naturels. L’imperméabilisation des surfaces (voiries, quais, aires de manœuvre) est connu pour favoriser le ruissellement des eaux pluviales, l’érosion des sols et la perturbation des cycles hydriques locaux. De plus, ces infrastructures peuvent être mal intégrées dans des continuités écologiques (trames vertes et bleues), fragmentant les habitats faunistiques et floristiques .

        Cette dégradation est particulièrement problématique lorsque les sites sont situés à proximité de zones Natura 2000, de ZNIEFF ou de réservoirs de biodiversité, malgré les mesures pour limiter leur installations. L'arbitrage est à mon sens bien souvent défavorable à la nature au vu des installations déjà présentes sur le territoire.\\
        \hline

        Transport et émissions indirectes & Le fait est, que certaines régions restent sous-équipées, les obligeant à exporter leurs déchets vers d’autres départements ou pays, transférant ainsi les impacts environnementaux vers des territoires tiers. Le tout entraînant des trajets plus longs pour la collecte et le transport des déchets. Cela génère un surcroît d’émissions de gaz à effet de serre et de polluants atmosphériques liés à la logistique\\
        \hline

        \end{longtable}

        Du point de vue environnemental, tout du moins, il me semble que les politiques publiques ont commencées à prendre en considération les impacts environnementaux des installations de gestion des déchets, mais leur application dépend encore des administrations locales et du bénéfice qu'elles pensent en tirer.
        
    \subsection{La gestion des déchets, de l'or dans nos poubelles ?}

    Au-delà de leur simple utilité, tous les équipements permettent de générer des retombées économiques, sociales et environnementales positives, faisant de la gestion des déchets un levier de la transition écologique.  

        \subsubsection{Les déchets ? Un or "vert"}
        
        Sur le plan environnemental, les infrastructures chargées de traiter et de valoriser les déchets se révèlent indispensables à la réduction de notre empreinte écologique. D’abord, les centres de tri détournent chaque année une part importante des déchets du circuit classique des ordures ménagères. Grâce à un contrôle méthodique des différents flux - papiers, plastiques, verre, métaux, etc. - ces installations facilitent le recyclage des matériaux et limitent le volume qui pourrait, partant à la décharge, engendrer des émissions de méthane, ou, allumé sur un grill, produire du dioxyde de carbone et d'autres polluants. En prévenant à la fois l'enfouissement et l'incinération, ils contribuent à la sauvegarde de la qualité de l'air et, plus largement, à l'atténuation du changement climatique. Par ailleurs, en restituant les matières déjà utilisées, on allège la pression sur les ressources naturelles - forêts, minerais ou encore pétrole - habituellement requises pour fabriquer des objets neufs, encourageant ainsi une économie circulaire et une gestion plus durable des matières premières.\\

        En outre, les équipements de valorisation énergétique permettent d’exploiter des déchets dits résiduels, c’est-à-dire ceux qui ne sont pas recyclables. Ces installations transforment les déchets en énergie, généralement de la chaleur, de l’électricité, ou les deux. Cette énergie est valorisée en tant que telle localement – par exemple, pour chauffer des bâtiments publics ou des logements – ou injectée dans les réseaux existants. Cela permet d’une part de réduire la dépendance à des sources d’énergie fossile au fort impact carbone, souvent importées, et d’autre part de renforcer l’autonomie énergétique des territoires, en valorisant des ressources locales renouvelables issues des déchets.\\

        De même, les installations de compostage ainsi que les unités de méthanisation traitent les déchets organiques (restes de repas, déchets verts, boues de stations d’épuration, etc.) pour les transformer en compost ou en biogaz. Le compost, à très forte valeur organique, sert d’amendement naturel pour les sols agricoles ou les espaces verts, substituant ainsi les engrais chimiques polluants et très coûteux à produire. Le biogaz est valorisé en chaleur, en électricité ou par injection dans les réseaux de gaz pour un mix énergétique plus propre, plus proche.

        \subsubsection{Les déchets ? Un levier économique pour les collectivités ? }

        Les sites de gestion des déchets constituent des pièces maîtresses dans l’édification d’un secteur économique en pleine émergence dont la dénomination a été renouvelée : « économie verte ». En participant à la transition vers un modèle durable, ils n’assument pas que leur vocation environnementale, ils peuvent agir également comme de véritables entrepreneurs économiques, qui créent de la valeur et du dynamisme territorial.\\

        Pour commencer, ils engendrent de très nombreux emplois directs (dans les domaines de la collecte des déchets, du tri manuel ou automatisé, de la maintenance des équipements, de la valorisation des matières) auprès de catégories de travailleurs très diverses allant des profils non qualifiés au technicien ou à l’ingénieur, incarnant ainsi un véritable secteur inclusif. Mais les sites de traitement des déchets induisent aussi une création d’emplois indirects dans des secteurs proches : logistique, transport, ingénierie environnementale, recherche et développement – conception de nouveaux processus de tri ou de valorisation. L’ensemble de cette chaîne participe à l’animation de l’économie locale, à l’innovation.\\

        Par ailleurs, les opérations de valorisation des déchets sont également une source de revenus pour les collectivités locales et les entreprises concernées. La mise en vente de matières premières secondaires recyclées (plastes, métaux, papiers, etc.), l’énergie (électricité, chaleur ou biogaz) générée, ou le compost pour agricultures commerciales comme produits différents de valorisation viennent donc rentabiliser le traitement des déchets pour réalimenter le coût global de gestion en réduisant certaines dépenses en enfouissement ou incinération, ou en élaborant des modèles économiques rentables et pérennes.\\

        Ensuite, cette mise en valeur peut être déployée avec le soutien public national ou européen, de ce que le développement durable, la lutte contre le changement climatique ou l’économie circulaire soutiennent. C’est en effet par des subventions, des exonérations fiscales ou des appels à projets que l’on peut faire avancer le financement d’équipements plus performants et innovants au profit du secteur économique.\\

    Pour conclure, à mon avis, les conséquences de l'installation d'un équipement quelconque de gestion des déchets restent encore bien trop importantes négativement à l'échelle locale. Cependant, il convient de reconnaître que l'expression familiale "les déchets des uns font le bonheur des autres"  est souvent juste.

    \subsection{L'implantation des installations de gestion de déchet - Conclusion}

    L’implantation des installations de gestion des déchets, bien qu’encadrée par des impératifs techniques et réglementaires, soulève d’importants enjeux territoriaux, environnementaux et sociaux. Dans un contexte de transition vers une économie circulaire, promue notamment par la loi \gls{AGEC} de 2020, la multiplication des infrastructures de tri, de traitement et de valorisation nécessite une approche intégrée de l’aménagement du territoire.\\

    La diversité des typologies d’installations (tri, incinération, stockage, valorisation énergétique, etc.) reflète une organisation spatiale fondée sur la proximité des gisements de déchets, l’accessibilité logistique, et la compatibilité avec les documents d’urbanisme. Toutefois, cette répartition est loin d’être équitable : certaines zones industrielles déjà équipées concentrent les infrastructures, tandis que les espaces résidentiels, en particulier les plus denses, opposent une résistance croissante à leur implantation. Ce déséquilibre est accentué par les inégalités socio-économiques, la saturation foncière, ou encore les pratiques administratives variables d’une DREAL à l’autre.\\

    Ces logiques d’implantation engendrent des impacts différenciés. Sur le plan social, les populations les plus vulnérables sont souvent les plus exposées aux nuisances (bruit, odeurs, pollution atmosphérique, trafic). Sur le plan environnemental, l’implantation de ces installations peut provoquer l’artificialisation des sols, la fragmentation des milieux naturels ou encore des pressions sur la ressource en eau, notamment lorsqu’elles sont situées à proximité de zones protégées comme les  \gls{ZNIEFF} ou les sites Natura 2000.\\

    Il est important de noter que, même si l'instruction des dossiers \gls{ICPE} est régie par une procédure nationale, elle reste largement influencée par les contextes locaux. Les préférences des préfets, les ressources des DREAL, et la culture administrative jouent un rôle clé dans les décisions, ce qui peut créer des inégalités dans le traitement des dossiers selon les territoires.\\

    En résumé, les installations de gestion des déchets ne peuvent pas être envisagées uniquement sous un angle technique ou réglementaire. Il est essentiel de considérer leur acceptabilité sociale, leur intégration dans le territoire, et leur impact sur l'environnement dans une approche systémique. Cela nécessite une meilleure coordination entre les différents acteurs, une planification spatiale juste, et une attention accrue aux enjeux de justice environnementale.



    

    
    \newpage
\section{Conclusion}

Pour conclure, il est évident que la réglementation concernant les Installations Classées pour la Protection de l’Environnement (\gls{ICPE}) représente un outil crucial dans la régulation de l'expansion des installations de gestion des déchets. Elle fournit un cadre structuré qui garantit une considération minutieuse des défis environnementaux, sanitaires et technologiques liés à ces actions. L'objectif principal de ce cadre réglementaire est d'éviter les impacts négatifs que ces équipements peuvent causer sur l'environnement naturel, la santé des individus et la sûreté publique. Ainsi, il occupe une position cruciale dans la gestion des risques, tant en amont qu'en aval des projets.\\

Sur la base d’une appréciation au cas par cas du risque présenté par chacune des installations, le régime \gls{ICPE} adosse un classement des activités à trois types de procédures : la déclaration, l’enregistrement et l’autorisation. Ces procédures correspondent à des niveaux d’exigences administratives et techniques croissants, en fonction de la dangerosité ou des effets ou impacts de l’activité exercée. Elles permettent d’adapter le contrôle des projets et le suivi des installations, en fonction de différents critères : la catégorie des substances utilisées, les risques liés aux matériaux, les capacités, les techniques mises en œuvre...\\ 

Ce régime s’appuie sur une importante nomenclature nationale regroupant près de 200 rubriques, qui classent les activités en fonction de leur technologie et des effets ou impacts qu’elles peuvent avoir sur l’environnement ou la santé. Chaque rubrique précise des critères et des seuils à partir desquels l’activité est soumise à l’un des trois régimes \gls{ICPE}. La nomenclature est au cœur du dispositif, en permettant un classement des activités à risques uniforme sur tout le territoire. Elle permet d’une part d’identifier la procédure à laquelle l’installation doit être soumise, d’autre part de définir le cadre technique et environnemental qui lui sera imposé. À ce titre, le dispositif réglementaire \gls{ICPE} garantit une stabilité, une lisibilité et une sécurité juridique des projets comme pour l’administration.\\

Cette certaine capacité à adapter les obligations réglementaires aux particularités de chaque installation donne à la réglementation \gls{ICPE} une flexibilité cruciale pour encadrer la diversité croissante des activités dans le domaine de la gestion des déchets. Elle ne se contente pas de renforcer la sécurité environnementale, mais elle aide également les porteurs de projets à naviguer dans un processus de conformité qui leur est propre. En ce sens, la réglementation \gls{ICPE} se présente comme une base solide et évolutive, guidant l’implantation et l’exploitation des installations tout en établissant un cadre clair pour minimiser leurs impacts sur l’environnement et les populations.\\

Néanmoins, un certain nombre de tensions structurelles traversent le dispositif \gls{ICPE} et sont mises en évidence dans la pratique. La complexité des procédures, la lourdeur administrative ainsi que l’hétérogénéité des pratiques entre territoires peuvent constituer des freins au développement de projets, notamment pour les structures de taille modeste ou les initiatives locales relevant de l’économie circulaire émergente. Un point notable réside dans le fonctionnement varié des Directions Régionales de l’Environnement, de l’Aménagement et du Logement (DREAL), responsables de l’instruction et du suivi des dossiers \gls{ICPE}. Selon les régions, les charges de travail, les priorités locales ou encore les interprétations spécifiques des textes réglementaires, les exigences imposées aux porteurs de projets peuvent varier considérablement, entraînant ainsi des disparités dans les conditions de mise en œuvre des installations.\\

Il existe d'importantes divergences dans les documents d'urbanisme locaux, comme les Plans Locaux d'Urbanisme (\gls{PLU}), les Schémas de Cohérence Territoriale (\gls{SCoT}) et les règlements de zones protégées. Ces différences peuvent avoir un impact significatif sur la compatibilité des projets avec les usages du sol et les règles d’aménagement. Parfois, une installation qui respecte toutes les normes techniques peut se heurter à des contraintes d’urbanisme imprévues, voire à des interdictions d’implantation dues à des zonages spécifiques ou à des servitudes d’utilité publique.\\

De plus, les contextes locaux de protection de l’environnement, comme la présence de zones naturelles protégées (telles que les \gls{ZNIEFF}, les sites Natura 2000 ou les zones humides) et les contraintes liées à la gestion de l’eau (via les \gls{SDAGE}, \gls{SAGE} ou les périmètres de protection de captages), compliquent encore plus l'examen des dossiers \gls{ICPE}. Ces facteurs exigent une attention particulière et souvent nuancée aux enjeux territoriaux, ce qui peut accentuer les différences de traitement entre des projets similaires situés dans des zones géographiques variées.\\

Bien que la réglementation \gls{ICPE} soit fondée sur une logique d’harmonisation à l’échelle nationale, sa mise en œuvre sur le terrain met en lumière une variété de pratiques et de contraintes locales. Cela soulève des questions sur sa capacité à assurer une transition écologique qui soit à la fois efficace, équitable et cohérente sur l’ensemble du territoire. Pour remédier à ces disparités, il est essentiel d'améliorer la coordination entre les différents services, de clarifier les exigences réglementaires et de simplifier les procédures. Cela rendrait le dispositif plus compréhensible, plus accessible et mieux adapté aux réalités locales.\\

Par ailleurs, la question de l’acceptabilité territoriale des installations de gestion des déchets demeure un enjeu central. La concentration de ces infrastructures dans des zones historiquement ou socialement défavorisées, conjuguée à un manque de concertation avec les populations concernées, alimente des inégalités d’exposition aux nuisances. Cette réalité met en évidence la nécessité de mieux articuler la réglementation \gls{ICPE} avec des logiques de justice environnementale et de cohésion territoriale, afin que la charge des efforts environnementaux ne pèse pas uniquement sur les territoires déjà fragilisés.\\

Il est important de noter qu'au-delà des contraintes techniques, juridiques et territoriales, la gestion des déchets constitue également une opportunité considérable, tant sur le plan économique que environnemental. Autrefois considérés comme de simples débris à se débarrasser, les déchets sont maintenant valorisés en tant que ressources potentielles dans une optique de valorisation matérielle. S'ils sont triés, traités et intégrés de manière appropriée dans des cycles de réutilisation ou de recyclage, ils peuvent constituer un atout stratégique pour la transition écologique. La transformation du "déchet" en "ressource" nécessite donc de dépasser une perspective purement administrative pour considérer la filière comme un domaine d'innovation, de génération de valeur et d'emplois durables – sous réserve que le cadre réglementaire et les mouvements territoriaux réussissent à soutenir ce changement.\\

En définitive, la réglementation \gls{ICPE}, bien qu’imparfaite dans son application, reste un outil fondamental de maîtrise des risques et d’encadrement des projets à fort potentiel impactant. Son renforcement, son adaptation aux dynamiques locales, ainsi que la montée en compétence des acteurs publics et privés, apparaissent comme des conditions nécessaires à l’accompagnement d’un développement plus durable, plus équilibré et plus résilient des filières de gestion des déchets.


    \appendix
    
\section{Annexe 1 - Table des figures}

     \begin{figure}[H]
        \includegraphics[width=200]{Images/Incendie.jpg}
        \centering
        \caption*{Figure 1: Incendie du centre de tri des collectes sélectives du Syctom : Face au Risque 7 avril 2025}
    \end{figure}

    \begin{figure}[H]
        \includegraphics[width=200]{Images/Banque mondiale.png}
        \centering
        \caption*{Figure 2: Graphique de la production de déchets projetée 2010-2100 par régions de la planète : Marcelo Pires Negrao 2017}
    \end{figure}

    \begin{figure}[H]
        \includegraphics[width=200]{Images/Gas effet de serre.jpg}
        \centering
        \caption*{Figure 3: Emissions de gaz à effet de serre par secteur en France sur la période 1990-2013 : Ministère de l'amènagement du territoire et de la transition écologique 2015}
    \end{figure}

    \begin{figure}[H]
        \includegraphics[width=150]{Images/Structure.png}
        \centering
        \caption*{Figure 4: Nomenclature ICPE : AIDA Ineris 2020}
    \end{figure}

    \begin{figure}[H]
        \includegraphics[width=200]{Images/Cerfa.png}
        \centering
        \caption*{Figure 5: Cerfa n°15271 (Déclaration initiale d'une installation classée) : Entreprendre Service Public 2025}
    \end{figure}

    \begin{figure}[H]
        \includegraphics[width=200]{Images/SEVESO.jpg}
        \centering
        \caption*{Figure 6: Les sites Seveso en France : Officiel Prévention 2020}
    \end{figure}

    \begin{figure}[H]
        \includegraphics[width=200]{Images/Rubrique.png}
        \centering
        \caption*{Figure 7: Rubrique 2712 et régime ICPE : AIDA Ineris 2025}
    \end{figure}

    \begin{figure}[H]
        \includegraphics[width=200]{Images/Arrêté.png}
        \centering
        \caption*{Figure 8: Extrait de l'article 5 de l'arrêté du 22/12/23 : AIDA Ineris 2025}
    \end{figure}

    \begin{figure}[H]
        \includegraphics[width=200]{Images/Cadastre.png}
        \centering
        \caption*{Figure 9: Analyse des abords de mon champs de pommes de terre : CRETEUR Tom 2025}
    \end{figure}

    \begin{figure}[H]
        \includegraphics[width=200]{Images/Localisation patates.png}
        \centering
        \caption*{Figure 10: Localisation de mon champs de patates : CRETEUR Tom 2025}
    \end{figure}

    \begin{figure}[H]
        \includegraphics[width=200]{Images/Patates.png}
        \centering
        \caption*{Figure 11: Compatibilité de mon champs de patates avec le PLU : CRETEUR Tom 2025}
    \end{figure}

    \begin{figure}[H]
        \includegraphics[width=200]{Images/Géorisques.png}
        \centering
        \caption*{Figure 12: Rapport de risques pour mon champs de pommes de terre : CRETEUR Tom 2025}
    \end{figure}

    \begin{figure}[H]
        \includegraphics[width=200]{Images/Natura 2000.png}
        \centering
        \caption*{Figure 13: Sites Natura 2000 à proximité de mon quartier : CRETEUR Tom 2025}
    \end{figure}

    \begin{figure}[H]
        \includegraphics[width=200]{Images/Incinération.jpg}
        \centering
        \caption*{Figure 14: Cartographie des UIOM en France : ITOM 2014}
    \end{figure}

    \begin{figure}[H]
        \includegraphics[width=200]{Images/Déchets dangereux.png}
        \centering
        \caption*{Figure 15: Localisation des installations de traitement des déchets dangereux régionaux en 2022 : Observatoire Régional des Déchets de Provences-Alpes-Côte d'Azur 2023}
    \end{figure}
    \section{Annexe 2 - Sitographie}
\sloppy
\begin{itemize}
    \item \textbf{{ADEME} – Agence de la transition écologique} : \href{https://www.{ADEME}.fr}{https://www.{ADEME}.fr}
    \item \textbf{{ADEME} – Bilan national du recyclage : des résultats mitigés suivant les filières} : \href{https://infos.{ADEME}.fr/economie-circulaire-dechets/2024/bilan-national-du-recyclage-des-resultats-mitiges-suivant-les-filieres/}{https://infos.{ADEME}.fr/economie-circulaire-dechets/2024/bilan-national-du-recyclage-des-resultats-mitiges-suivant-les-filieres/}
    \item \textbf{{ADEME} – Chiffres clés des déchets 2021 (PDF)} : \href{https://librairie.{ADEME}.fr/cadic/7054/chiffres-cles-des-dechets-edition-2021.pdf}{https://librairie.{ADEME}.fr/cadic/7054/chiffres-cles-des-dechets-edition-2021.pdf}
    \item \textbf{{ADEME} – Dossiers déchets et recyclage} : \href{https://www.{ADEME}.fr/expertises/dechets}{https://www.{ADEME}.fr/expertises/dechets}
    \item \textbf{{ADEME} – Économie circulaire et territoires} : \href{https://librairie.{ADEME}.fr/dechets-economie-circulaire/6053-l-economie-circulaire-dans-les-territoires.html}{https://librairie.{ADEME}.fr/dechets-economie-circulaire/6053-l-economie-circulaire-dans-les-territoires.html}
    \item \textbf{{ADEME} – Outils pour l’évaluation environnementale des installations de gestion des déchets (2022)} : \href{https://librairie.{ADEME}.fr}{https://librairie.{ADEME}.fr}
    \item \textbf{{ADEME} – Planification territoriale des installations de gestion des déchets (2023)} : \href{https://librairie.{ADEME}.fr}{https://librairie.{ADEME}.fr}
    \item \textbf{La loi anti-gaspillage pour une économie circulaire (AGEC} : \href{https://www.ecologie.gouv.fr/loi-anti-gaspillage-economie-circulaire}{https://www.ecologie.gouv.fr/loi-anti-gaspillage-economie-circulaire}
    \item \textbf{AIDA – Base réglementaire environnement INERIS} : \href{https://aida.ineris.fr}{https://aida.ineris.fr}
    \item \textbf{Arrêté du 26 mars 2012 relatif aux installations soumises à enregistrement (rubrique 2710)} : \href{https://aida.ineris.fr}{https://aida.ineris.fr}
    \item \textbf{Base des arrêtés ministériels ICPE (via AIDA – Ineris)} : \href{https://aida.ineris.fr}{https://aida.ineris.fr}
    \item \textbf{Cadastre} : \href{https://cadastre.gouv.fr/}{https://cadastre.gouv.fr/}
    \item \textbf{CIREGE – Université de Reims : La territorialisation de l’économie circulaire} : \href{https://www.univ-reims.fr}{https://www.univ-reims.fr}
    \item \textbf{Syctom : La Centre de tri à Paris XVII} : \href{https://https://www.syctom-paris.fr/centres-de-traitement/centres-de-tri/paris-xvii.html}{https://https://www.syctom-paris.fr/centres-de-traitement/centres-de-tri/paris-xvii.html}
    \item \textbf{CITEPA – Données d’émissions GES} : \href{https://www.citepa.org}{https://www.citepa.org}
    \item \textbf{Code de l’environnement (notamment articles L512-1 à L512-8)} : \href{https://www.legifrance.gouv.fr}{https://www.legifrance.gouv.fr}
    \item \textbf{Commissariat Général au Développement Durable (CGDD) – Inégalités environnementales et sociales (2019)} : \href{https://www.statistiques.developpement-durable.gouv.fr/inegalites-environnementales-et-sociales}{https://www.statistiques.developpement-durable.gouv.fr/inegalites-environnementales-et-sociales}
    \item \textbf{DGPR – Guide méthodologique pour la prise en compte des impacts environnementaux ICPE (2021)} : \href{https://www.ecologie.gouv.fr/icpe}{https://www.ecologie.gouv.fr/icpe}
    \item \textbf{DONNEES ET ETUDES STATISTIQUES – Émissions de gaz à effet de serre par secteur} : \href{https://www.donnees.statistiques.developpement-durable.gouv.fr/lesessentiels/climat/climat-effet-serre-secteur-france.htm}{https://www.donnees.statistiques.developpement-durable.gouv.fr/lesessentiels/climat/climat-effet-serre-secteur-france.htm}
    \item \textbf{DONNEES ET ETUDES STATISTIQUES – Émissions de gaz à effet de serre par secteur} : 
    \href{https://entreprendre.service-public.fr/vosdroits/R42639}{https://entreprendre.service-public.fr/vosdroits/R42639}
    \item \textbf{FERME SOLAIRE – ICPE et déchets : comprendre la règlementation} : \href{https://www.fermesolaire.fr/magazine/icpe-et-dechets-comprendre-la-reglementation-pour-les-activites-de-gestion-des-dechets}{https://www.fermesolaire.fr/magazine/icpe-et-dechets-comprendre-la-reglementation-pour-les-activites-de-gestion-des-dechets}
    \item \textbf{France Stratégie – Économie circulaire et recyclage} : \href{https://www.strategie.gouv.fr}{https://www.strategie.gouv.fr}
    \item \textbf{France Stratégie – Souveraineté industrielle et économie circulaire} : \href{https://www.strategie.gouv.fr/publications/economie-circulaire-souverainete-industrielle}{https://www.strategie.gouv.fr/publications/economie-circulaire-souverainete-industrielle}
    \item \textbf{Geoportail} : \href{https://www.geoportail.gouv.fr/}{https://www.geoportail.gouv.fr/}
    \item \textbf{Georisques – Registre des ICPE} : \href{https://www.georisques.gouv.fr/dossiers/installations-classees}{https://www.georisques.gouv.fr/dossiers/installations-classees}
    \item \textbf{INERIS – Base de données ICPE (2023)} : \href{https://www.ineris.fr}{https://www.ineris.fr}
    \item \textbf{INERIS – Fiches risques et retours d’expérience incendies (2023)} : \href{https://www.ineris.fr}{https://www.ineris.fr}
    \item \textbf{INERIS – ICPE et risques industriels} : \href{https://www.ineris.fr/fr/activites/installations-classees}{https://www.ineris.fr/fr/activites/installations-classees}
    \item \textbf{INSEE – Données régionales sur les déchets et l’environnement} : \href{https://www.insee.fr/fr/statistiques/4277619}{https://www.insee.fr/fr/statistiques/4277619}
    \item \textbf{Ladocumentationfrancaise.fr – Rapports publics (IGEDD, etc.)} : \href{https://www.ladocumentationfrancaise.fr}{https://www.ladocumentationfrancaise.fr}
    \item \textbf{LE MONDE – Où sont situés les 1 300 sites Seveso en France} : 
    \href{https://www.lemonde.fr/les-decodeurs/article/2022/08/03/ou-sont-situes-les-1-300-sites-seveso-en-france-et-quels-sont-les-risques_6137057_4355770.html}{https://www.lemonde.fr/les-decodeurs/article/2022/08/03/ou-sont-situes-les-1-300-sites-seveso-en-france-et-quels-sont-les-risques_6137057_4355770.html}
    \item \textbf{Légifrance – Code de l’environnement (partie réglementaire – Livre V)} : \href{https://www.legifrance.gouv.fr/codes/texte_lc/LEGITEXT000006074220}{https://www.legifrance.gouv.fr/codes/texte_lc/LEGITEXT000006074220}
    \item \textbf{Légifrance – Nomenclature ICPE} : \href{https://www.legifrance.gouv.fr}{https://www.legifrance.gouv.fr}
    \item \textbf{LOI n° 2020-105 du 10 février 2020 (loi AGEC)} : \href{https://www.legifrance.gouv.fr}{https://www.legifrance.gouv.fr}
    \item \textbf{MILLENAIRE 3 – Le recyclage, quelle juste place pour cette pratique ?} : \href{https://millenaire3.grandlyon.com/ressources/2022/le-recyclage-quelle-juste-place-pour-cette-pratique?}{https://millenaire3.grandlyon.com/ressources/2022/le-recyclage-quelle-juste-place-pour-cette-pratique?}
    \item \textbf{MINISTÈRE DE LA TRANSITION ÉCOLOGIQUE – Dossier ICPE} : \href{https://www.ecologie.gouv.fr/icpe}{https://www.ecologie.gouv.fr/icpe}
    \item \textbf{MINISTÈRE DE LA TRANSITION ÉCOLOGIQUE – Gestion des déchets} : \href{https://www.ecologie.gouv.fr/gestion-des-dechets}{https://www.ecologie.gouv.fr/gestion-des-dechets}
    \item \textbf{MINISTÈRE DE LA TRANSITION ÉCOLOGIQUE – ICPE et déchets} : \href{https://www.ecologie.gouv.fr/installations-classees-protection-lenvironnement-icpe}{https://www.ecologie.gouv.fr/installations-classees-protection-lenvironnement-icpe}
    \item \textbf{MINISTÈRE DE LA TRANSITION ÉCOLOGIQUE – Plan national de prévention des déchets} : \href{https://www.ecologie.gouv.fr/plan-national-prevention-des-dechets}{https://www.ecologie.gouv.fr/plan-national-prevention-des-dechets}
    \item \textbf{OBSERVATOIRE DES TERRITOIRES – Centres de tri des déchets} : \href{https://www.observatoire-des-territoires.gouv.fr/nombre-de-centres-de-tri-de-dechets-menagers-et-assimiles-dma-}{https://www.observatoire-des-territoires.gouv.fr/nombre-de-centres-de-tri-de-dechets-menagers-et-assimiles-dma-}
    \item \textbf{OFFICIEL PREVENTION – Prévention des risques industriels des sites classés Seveso} : \href{https://www.officiel-prevention.com/dossier/protections-collectives-organisation-ergonomie/risque-chimique-2/prevention-des-risques-industriels-des-sites-classes-seveso}{https://https://www.officiel-prevention.com/dossier/protections-collectives-organisation-ergonomie/risque-chimique-2/prevention-des-risques-industriels-des-sites-classes-seveso}
    \item \textbf{ORDIF – Observatoire régional des déchets d’Île-de-France} : \href{https://www.ordif.com}{https://www.ordif.com}
    \item \textbf{Plan Local d’Urbanisme - Mairie de Saint-Romain d’Ay} : \href{https://www.stromainday.fr/plan-local-d-urbanisme/}{https://www.stromainday.fr/plan-local-d-urbanisme/}
    \item \textbf{POLLUTEC Learn and Connect 365 – Nouveaux objectifs de recyclage} : \href{https://learnandconnect.pollutec.com/de-nouveaux-objectifs-de-recyclage-en-europe/}{https://learnandconnect.pollutec.com/de-nouveaux-objectifs-de-recyclage-en-europe/}
    \item \textbf{RECYCLAGE RECUPERATION – Déchets inertes et ICPE} : \href{https://www.recyclage-recuperation.fr/archives-dechets-com/le-stockage-des-dechets-inertes-releve-desormais-du-regime-icpe}{https://www.recyclage-recuperation.fr/archives-dechets-com/le-stockage-des-dechets-inertes-releve-desormais-du-regime-icpe}
    \item \textbf{RESEARCH GATE – Production projetée de déchets par région} : \\
    \href{https://www.researchgate.net/figure/Graphique-production-de-dechets-projetee-2010-2100-par-regions-de-la-planete-Source-des_fig3_324690242}{https://www.researchgate.net/figure/Graphique-production-de-dechets-projetee-2010-2100-par-regions-de-la-planete-Source-des_fig3_324690242}
    \item \textbf{RTBF ACTUS – Incendie dans le zoning de Rochefort} : \href{https://www.rtbf.be/article/incendie-dans-le-zoning-de-rochefort-300-tonnes-de-dechets-et-un-hangar-consumes-pas-de-degagements-anormalement-toxiques-dans-les-fumees-11001609}{https://www.rtbf.be/article/incendie-dans-le-zoning-de-rochefort-300-tonnes-de-dechets-et-un-hangar-consumes-pas-de-degagements-anormalement-toxiques-dans-les-fumees-11001609}
    \item \textbf{WIKIPEDIA – Gestion des déchets en France} : \href{https://fr.wikipedia.org/wikiGestion_des_d%C3%A9chets_en_France}{https://fr.wikipedia.org/wiki/Gestion_des_d%C3%A9chets_en_France}
\end{itemize}



    \section{Annexe 3 - Bibliographie}

ADEME (2022). Déchets chiffres-clés – Édition 2022. Agence de la Transition Écologique.\\

Barraqué, B., & Larrue, C. (2021). Justice environnementale et politiques locales de l'environnement. L'Harmattan.\\

CITEPA (2023). Inventaire des émissions de gaz à effet de serre de la France. Centre interprofessionnel technique d'études de la pollution atmosphérique.\\

Haut Conseil pour le Climat (2022). Maîtriser l’empreinte carbone de la France : une ambition climatique pour tous. Rapport annuel.\\

IGEDD – Inspection Générale de l'Environnement et du Développement Durable (2022). Évaluation de la mise en œuvre des procédures d’autorisation environnementale. Ministère de la Transition écologique.\\

Ministère de la Transition Écologique (2023). Bilan des incendies sur les installations de déchets et ICPE en France. Direction Générale de la Prévention des Risques (DGPR).\\

Moulis, C. (2019). Les installations classées pour la protection de l’environnement : un droit à la croisée des chemins. Revue juridique de l’environnement, 44(2).\\

Code de l’environnement, Partie législative et réglementaire, Livre V : Prévention des pollutions, des risques et des nuisances.
Éditions Dalloz, mise à jour annuelle.
Référence pour les articles L511-1 à L517-2.\\

Ministère de la Transition écologique – Direction Générale de la Prévention des Risques (DGPR) (2023).
Chiffres clés des ICPE en France.
Paris, Ministère de la Transition écologique.\\

IGEDD (Inspection Générale de l’Environnement et du Développement Durable) (2022).
Évaluation du régime d’autorisation environnementale.
La Documentation Française. Rapport public.\\

INERIS (Institut National de l’Environnement Industriel et des Risques) (2021).
Cadre réglementaire des installations classées : approche par les risques industriels.
Rapport technique, Verneuil-en-Halatte.\\

ADEME (2021). Chiffres-clés des déchets – Édition 2021. Agence de la transition écologique.\\

France Stratégie (2021). Souveraineté industrielle et économie circulaire : défis et opportunités. Paris : La Documentation Française.\\

Coutard, O., Lorrain, D. (2011). Les réseaux territoriaux de l'économie circulaire. Revue d’Économie Régionale et Urbaine, n°5.\\

Le Guen, R., (2019). Inégalités socio-environnementales en France : état des lieux et perspectives. Commissariat Général au Développement Durable (CGDD).\\

Durand, A. (2020). Justice environnementale et gouvernance territoriale des déchets. Paris : L’Harmattan.\\

Laurent, É. (2011). Le nouvel âge de l’inégalité : économie et politique dans une société en transition. Éditions du Seuil.\\

Barles, S. (2007). Urban metabolism and recycling policies in Paris: History and current issues. Journal of Industrial Ecology, 11(2), 5–15.\\

CIREGE (2020). La territorialisation des enjeux de l’économie circulaire : une lecture critique. Université de Reims.\\

Vernier, J (1971). La bataille de l'environnement\\

Tzu, S (-500 env). L'art de la Guerre\\

Faure, E (1966). Prévoir le présent\\ 
    \section{Annexe 4 - Responsabilité sociale et environnementale}

        Lors de mon stage, mes impacts environnementaux directs ont été de plusieurs natures. L’estimation exhaustive de ces impacts en termes de potentiel de réchauffement climatique est détaillé ci-dessous.

            Au cours de mon stage, je me suis rendu sur le lieu d’accueil 5 jours par semaine pendant 19 semaines, soit un total de 95 jours. Les trajets ont été effectués en voiture pour 95 jours, sur une distance moyenne de 39 km aller-retour. Selon la Base Carbone de l’\gls{ADEME}, l’empreinte carbone moyenne est estimée à 192 $\text{ g éq. } CO_2/passager.km$ pour la voiture. Sur cette base, l’impact carbone de mes déplacements peut être estimé comme suit :

            \begin{equation*}
                192 * 95 * 39 = 711,3 \text{ kg éq. } CO_2
            \end{equation*}

            Il est cependant important de souligner que, indépendamment du bilan carbone associé, je ne disposais d’aucune alternative de transport réellement applicable. Le recours à des plateformes externes aurait représenté une contrainte logistique importante, pour un bénéfice marginal. L’utilisation du vélo, bien qu’écologiquement avantageuse, n’était pas envisageable en raison de problèmes de santé actuels et de l’absence d’équipement adapté. \\
            
            Par ailleurs, il convient de rappeler que les valeurs issues de la Base Carbone sont assorties d’une incertitude estimée à 60\% (sans davantage de précisions sur le périmètre concerné), et qu’elles correspondent à des valeurs moyennes susceptibles de diverger sensiblement de la réalité observée. 
            Afin de réaliser ce stage, j’ai principalement utilisé un ordinateur portable allumé en continu durant mes horaires de travail. Ce constat augmente de l’impact environnemental imputable au stage, puisque les ressources mobilisées n'étaient pas réellement mutualisées. 
 
            Aucun autre impact direct n’a été relevé, puisque je ne me suis pas rendu sur le lieu d'accueil avant le démarrage du stage. Ma production de gaz à effet de serre est ainsi relativement faible, reste néanmoins significative au regard des 2 $\text{tonnes eq. }CO_2$ par personne en moyenne en France à l’horizon 2050, objectif fixé par la Stratégie Nationale Bas Carbone (2019) en conformité avec les Accords de Paris de 2015.\\
        
            Cependant, cette analyse présente plusieurs limites. Pour être pleinement exhaustive, il conviendrait d’intégrer l’impact environnemental lié à l’utilisation des réseaux de l’entreprise ainsi que celui des locaux. Par ailleurs, cette empreinte ne reflète en rien mes émissions annuelles personnelles, excluant notamment celles liées au logement ou aux déplacements privés. De ce fait, toute comparaison directe reste dénuée de sens sur les plans mathématiques et environnementaux.


 


\end{document}
