\newpage
\sloppy
\section{Introduction}

Qu'est-ce qu'un déchet ?\\

Jacques Vernier, dans son livre La bataille de l’environnement (1971), le définit comme :\\
\begin{center}
"Un produit dont personne ne veut à l’endroit où il se trouve."
\end{center}
C’est souvent un objet nuisible pour la vue, l’odorat ou la santé, et que la nature refuse de réintroduire dans son cycle.\\
Pour le préhistorien ou l’archéologue, le déchet est un trésor :
Un vase ébréché, le squelette d’un animal vieux de milliers d’années, un outil rouillé de l’époque sumérienne sont une mine d’informations pour la science.\\

Les chasseurs de manuscrits trouvent la majorité de leurs documents originaux en fouillant dans les déchets produits par les auteurs célèbres. C’est en apprenant que sa propre poubelle faisait régulièrement l’objet de telles investigations que Jean-Paul Sartre prit la décision radicale de détruire systématiquement tous les brouillons de ses écrits en les brûlant.\\

Le déchet, en tant que trace matérielle de l’activité humaine, est aussi ancien que les sociétés elles-mêmes. Dans le contexte contemporain, marqué par la surconsommation et la production massive de biens, il acquiert une signification nouvelle. Il peut, si l’on parvient à en tirer pleinement parti, devenir une ressource stratégique. Ainsi, les rebuts de la société de consommation pourraient-ils se transformer en véritables "cornes d’abondance ", pour peu que les conditions techniques, économiques et sociales soient réunies. Néanmoins, pour que ce potentiel se réalise, les installations dédiées à la gestion des déchets devront surmonter un ensemble de défis majeurs. Parmi ceux-ci figurent notamment l’innovation technologique, la mobilisation d’investissements conséquents, ainsi que la résilience face aux risques, y compris ceux liés à des catastrophes industrielles ou environnementales.\\

Le lundi 7 avril, à environ 19h45, un incendie a vite pris dans le centre de tri des collectes sélectives du Syctom, sis dans le 17e arrondissement de Paris. Malgré l’intervention rapide des agents d’exploitation et l’activation des dispositifs de sécurité opérationnels, les premières mesures n’ont pas permis de maîtriser le sinistre, occasionnant des dommages matériels conséquents. 

     \begin{figure}[H]
            \includegraphics[width=\textwidth]{Images/Incendie.jpg}
            \centering
            \caption{Incendie du centre de tri des collectes sélectives du Syctom : Face au Risque 7 avril 2025}
        \end{figure}


Ce type d’événement illustre les aléas inhérents à l’exploitation de telles installations vouées à une gestion des déchets. Ce n’est rien d’autre qu’une illustration de questions plus fondamentales qui relèvent de la sécurité industrielle, de la prévention des pollutions mais aussi de la maîtrise des nuisances. Ces installations, déclarées au titre des Installations Classées pour la Protection de l’Environnement (\gls{ICPE}) manipulent au quotidien des matières et substances dont le caractère pourrait être inflammable, toxique ou polluant.\\

En France, ce sont près de 500 centres de tri et plateformes de recyclage qui relèvent du régime \gls{ICPE}, compte tenu de la nature des matériaux traités et des impacts environnementaux potentiellement associés à leurs activités (\gls{ADEME}). La fréquence et la gravité des incidents rapportés soulignent l’importance d’une régulation rigoureuse, fondée sur la prévention des risques, le contrôle des installations et l’amélioration continue des pratiques dans le secteur du traitement et de la valorisation des déchets.\\

La réglementation relative aux \gls{ICPE}, instaurée par la loi du 19 juillet 1976, a pour objectif principal la prévention des nuisances et des risques susceptibles d’être générés par certaines activités industrielles et agricoles. Elle repose sur un encadrement administratif rigoureux, modulé en fonction du niveau de danger que présentent les installations concernées. Ce dispositif réglementaire s’inscrit aujourd’hui dans un contexte de transition vers une économie circulaire, qui constitue à la fois une priorité nationale et un engagement fort au niveau européen. Dans cette perspective, les infrastructures de recyclage occupent une position stratégique : elles contribuent à la réduction de l’enfouissement des déchets, à la limitation de la consommation de ressources primaires, ainsi qu’à la diminution des émissions de gaz à effet de serre.\\

Néanmoins, le développement de ces infrastructures soulève des enjeux sociaux et territoriaux importants. Parmi ceux-ci figurent l’opposition des riverains, la tendance à concentrer les installations dans des zones déjà marquées par des vulnérabilités socio-économiques, le déficit d’acceptabilité locale et l’inégalité d’accès aux bénéfices environnementaux générés par ces dispositifs. Ces problématiques témoignent de la complexité à concilier les impératifs environnementaux avec les dynamiques sociales et spatiales dans les territoires.\\

Dans ce contexte, alors que la France s’est engagée, par le biais de la loi relative à la lutte contre le gaspillage et à l’économie circulaire (loi \gls{AGEC}, 2020), à accroître significativement le nombre d’installations de valorisation matière pour atteindre un objectif de 65 \% de recyclage des déchets municipaux d’ici à 2035, le cadre réglementaire ICPE apparaît comme un outil ambivalent. D’une part, il constitue un levier essentiel pour sécuriser le développement de ces activités du point de vue environnemental. D’autre part, il met en lumière les tensions persistantes qui entourent l’implantation territoriale de ces infrastructures. Comme le souligne l’Agence Européenne pour l’Environnement (\gls{AEE}, 2022), les activités industrielles, y compris celles considérées comme « vertes », continuent de générer des formes de pollution localisée et des nuisances, touchant de manière disproportionnée les populations les plus vulnérables.\\

Ce rapport a pour but d’analyser de façon approfondie comment le cadre réglementaire des \gls{ICPE} encadre le développement des installations de gestion des déchets et, donc, dans quelle mesure ce cadre permet de prévenir et de circonscrire les impacts environnementaux et sociaux potentiellement engendrés par ces activités. La réglementation ICPE comme outil de prévention des risques industriel et environnemental est un dispositif clé au pilotage du développement de ces installations potentiellement impactantes.\\

La problématique est donc la suivante :\\

\textit{Comment la réglementation des ICPE permet-elle d’encadrer efficacement le développement des infrastructures de gestion des déchets, tout en limitant leurs impacts environnementaux et sociaux ?}\\

Dans un premier temps, nous reviendrons sur le rôle structurant de la réglementation \gls{ICPE} dans la gestion des risques inhérents aux activités de traitement et de valorisation des déchets, et nous verrons de quelle façon ce cadre réglementaire basé sur une approche fondée sur des principes de prévention, de précaution et de proportionnalité va encadrer le recyclage pour en limiter les effets nocifs pour la santé humaine et pour l’environnement. Nous prendrons également en compte les exigences techniques et organisationnelles afférentes aux exploitants au sein de ce cadre réglementaire, ainsi que les outils de contrôle et de rédaction mis en oeuvre par les administrations chargées de l’autorisation et du contrôle.\\

Dans une seconde analyse, nous nous interrogerons sur l’instruction et la constitution dans son ensemble d’un dossier \gls{ICPE} pour un projet de création ou de modification d’installation de recyclage. Nous mettrons ici en exergue les principales phases administratives et techniques, c’est-à-dire de la recherche du site jusqu’à l’obtention de l’autorisation ou à l’enregistrement, en passant par l’élaboration de l’étude d’impact, la concertation du public et l’évaluation des risques. L’objectif est ici de mettre en évidence une certaine complexité dans le processus ainsi qu’un certain nombre d’exigences spécifiques pour les porteurs de projet dans ce secteur.\\

Enfin, nous nous pencherons sur les enjeux socio-environnementaux liés au développement de ces installations sur le territoire. Nous analyserons en particulier les disparités d’exposition des populations aux nuisances générées par ces activités (bruit, odeurs, trafic, pollution, etc.) et les facteurs susceptibles de renforcer ou de limiter leur acceptabilité sociale. Cette dernière partie permettra d’éclairer les tensions potentielles entre impératifs écologiques, équité territoriale et dynamiques locales d’aménagement.\\
