\section{Annexe 4 - Responsabilité sociale et environnementale}

        Lors de mon stage, mes impacts environnementaux directs ont été de plusieurs natures. L’estimation exhaustive de ces impacts en termes de potentiel de réchauffement climatique est détaillé ci-dessous.

            Au cours de mon stage, je me suis rendu sur le lieu d’accueil 5 jours par semaine pendant 19 semaines, soit un total de 95 jours. Les trajets ont été effectués en voiture pour 95 jours, sur une distance moyenne de 39 km aller-retour. Selon la Base Carbone de l’\gls{ADEME}, l’empreinte carbone moyenne est estimée à 192 $\text{ g éq. } CO_2/passager.km$ pour la voiture. Sur cette base, l’impact carbone de mes déplacements peut être estimé comme suit :

            \begin{equation*}
                192 * 95 * 39 = 711,3 \text{ kg éq. } CO_2
            \end{equation*}

            Il est cependant important de souligner que, indépendamment du bilan carbone associé, je ne disposais d’aucune alternative de transport réellement applicable. Le recours à des plateformes externes aurait représenté une contrainte logistique importante, pour un bénéfice marginal. L’utilisation du vélo, bien qu’écologiquement avantageuse, n’était pas envisageable en raison de problèmes de santé actuels et de l’absence d’équipement adapté. \\
            
            Par ailleurs, il convient de rappeler que les valeurs issues de la Base Carbone sont assorties d’une incertitude estimée à 60\% (sans davantage de précisions sur le périmètre concerné), et qu’elles correspondent à des valeurs moyennes susceptibles de diverger sensiblement de la réalité observée. 
            Afin de réaliser ce stage, j’ai principalement utilisé un ordinateur portable allumé en continu durant mes horaires de travail. Ce constat augmente de l’impact environnemental imputable au stage, puisque les ressources mobilisées n'étaient pas réellement mutualisées. 
 
            Aucun autre impact direct n’a été relevé, puisque je ne me suis pas rendu sur le lieu d'accueil avant le démarrage du stage. Ma production de gaz à effet de serre est ainsi relativement faible, reste néanmoins significative au regard des 2 $\text{tonnes eq. }CO_2$ par personne en moyenne en France à l’horizon 2050, objectif fixé par la Stratégie Nationale Bas Carbone (2019) en conformité avec les Accords de Paris de 2015.\\
        
            Cependant, cette analyse présente plusieurs limites. Pour être pleinement exhaustive, il conviendrait d’intégrer l’impact environnemental lié à l’utilisation des réseaux de l’entreprise ainsi que celui des locaux. Par ailleurs, cette empreinte ne reflète en rien mes émissions annuelles personnelles, excluant notamment celles liées au logement ou aux déplacements privés. De ce fait, toute comparaison directe reste dénuée de sens sur les plans mathématiques et environnementaux.


